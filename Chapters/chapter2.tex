%!TEX root = ../template.tex
%%%%%%%%%%%%%%%%%%%%%%%%%%%%%%%%%%%%%%%%%%%%%%%%%%%%%%%%%%%%%%%%%%%%
%% chapter2.tex
%% NOVA thesis document file
%%
%% Chapter with the template manual
%%%%%%%%%%%%%%%%%%%%%%%%%%%%%%%%%%%%%%%%%%%%%%%%%%%%%%%%%%%%%%%%%%%%

\typeout{NT FILE chapter2.tex}

\chapter{Theoretical and Technical Concepts}
\label{cha:technical_concepts}

\section{Non-verbal Communication with Physiological Sensors: The Aesthetic Domain of Biosignals and Neural Networks}
\label{subsec:title}

% 1. Most basic definition
% 2. Some specific details
% 3. Relevance to Thesis
% 4. "This is described further in Sections a, b, and c

\subsection{Non-Verbal Communication}

Non-verbal communication is an umbrella term used to distinguish modalities of interpersonal exchange that are independent from speech, many of which are unconsciously transmismitted during social relations, this encompasses a variety of modalities that convey emotions, feelings, and messages. Behavioural analysis grounded in psychology research calls attention to the emotional information disclosed by non-verbal social cues, in particular, actions that are involuntary. Computing and Physiology researcher Alex Petland frames these as \textit{Honest Signals} to articulate a level of emotional authenticity while complimentary studies note the permitted degree of ambiguity.

\textit{“The unconscious quality of particular informative non-verbal behavioural cues grants a level of authenticity”} \fullcite{pentland_honest_2010} \cite{pentland_honest_2010}
% Pentland et al. 2010. Honest Signals: How They Shape Our World

\subsection{Physiological Sensors}

Physiological signals provide a measurement of biophysical, biomechanical and bioelectrical changes that occur from within the body. These are widely used to monitor a variety non-verbal social cues, measuring the ‘invisible’ signals that are otherwise not explicitly perceived. 

Non-verbal behaviour is commonly associated with “body-language”, aspects such as posture, gaze and other observable traits. But how about bodily signals that are invisible from a third-person perspective? Physiological sensors (or biosignals) are capable of monitoring these internal changes, that can be associated with an emotional response, typically in accordance to a measure of arousal, that operates on a linear scale.

\textit{“In other words, nonverbal behavioural cues are the physical, machine detectable evidence of affective phenomena not otherwise accessible to experience, an ideal point for technology and human sciences to meet.”}

\fullcite{vinciarelli_towards_2011} \cite{vinciarelli_towards_2011}
% Vinciarelli et al. ’Towards a Technology of Nonverbal Communication: Vocal Behavior in Social and Affective Phenomena’

% These signals are traditionally represented as a low-dimensional format. For example, one ECG sensor will pick up a uniform series of amplitudes over time, isolating the electrical response from the heart, separated from the surroundings

\subsection{The Aesthetic Domain}

The Aesthetics Domain, to put forward such a heavily loaded concept, we can reduce this to its literal derivative of \textit{aístesis}, which describes the process of perceiving through sensory engagement. Whilst this phenomenon is continually re-evaluated, it can be assumed to operate on a scale of order and complexity. This is essentially how we as humans are able to perceive complex information contained in our surroundings, reducing to a meaningful inference, through a process of learning as a result of personal experiences. From this, we consider ambiguity as an affordance for expressive exchange. In \textit{Two Modernist Approaches to Linking Art and Science}, Eric R. Kandel ties relevance to the art history concept of the beholder's share to the biological understanding of the human mind \cite{kandel_two_2013}.

\textit{"Human emotional life is rich; we can experience a huge number of emotions, possibly a continuum, not just a few for which we have words, like fear, sadness, joy, etc."}

\fullcite{perlovsky_aesthetic_2014} \cite{perlovsky_aesthetic_2014}

% Kris & Kaplan. 1952. Aesthetic Ambiguity

\subsection{Neural Networks}

An artificial neural network aims to simulate the core functions of the human brain, used to define complex input-output patterns, using previously learned information to comprehend sensory inputs. Modern Data Science practices have validated Neural Networks as an effective model for associating human-understandable contexts to ambiguous sensor data. Furthermore, applying an iterative, human-centred approach to augment aesthetic potentials in biofeedback contexts

Combined with the previous components, we consider the human-like qualities of Artificial Neural Networks as the technical foundation towards emotional engagement with physiological data. As Modern Data Science practices have already validated Neural Networks as an effective model for associating human-understandable contexts to ambiguous sensor data, we reflect upon human-centred augmentations by which the user is immersed in the learning process. 

\textit{"Treating embodied knowledge as something that cannot be accessed directly and only through examples of action (treating the learning algorithm as a “black box”) is therefore missing a lot.” }

\fullcite{gillies_understanding_2019} \cite{gillies_understanding_2019}
% M. Gillies, ‘Understanding the Role of Interactive Machine Learning in Movement Interaction Design’

% \section{Affective Computing}

\section{Physiological Signals}

\subsection{Physiological Signal Categories} 
\label{subsec:catagories}

In this document, we will cover a range of Physiological Signals in the context of potential interaction modalities. Physiological signals can be categorized according to the origin of the activity that is being recorded from the body \cite{enderle_introduction_2012}. The signals we'll be assessing and comparing in this work are defined as bioelectrical and biomechanical. bioelectrical signals provide a measurement of the electric and electromagnetic fields produced by living cells. Examples include Electromyography (EMG), Electrocardiogram (ECG), and Electrodermal activity (EDA) \cite{malmivuo_bioelectromagnetismprinciples_1995}. Biomechanical signals on the other hand measure the physical forces produced by or applied to the cells, tissues and organs. These include respiratory cycles and acceleration of the limbs \cite{guerreiro_bitalino_2013, pacelli_sensing_2006}. The complete list of categories based on anatomical origin is comprised of biomagnetic, biochemical, bioacoustic and biooptical signals.   

\subsection{Controllability of Signals}

In addition to the signal categories described in section \ref{subsec:catagories}, we'll also take into account the controllability a subject has on a given signal, dependant on the source, which can be classified as Voluntary, Indirect or Involuntary. Through Voluntary sources, the user can intentionally manipulate the signal with a high degree of freedom. For example, muscle contractions or join displacements, activities that are associated with the somatic nervous system. Indirect (or Mixed) sources grant the user partial control where as Involuntary sources indicate that there's almost no control over the outcome \cite{da_silva_biosignal_2017}. Involuntary sources are transmitted from the autonomic nervous system as they occur without conscious control \cite{lenman_human_1975}.

% \subsection{Electromyography}

% \subsection{Electrocardiography}

% \subsection{Electrodermal activity}

\section{Transdisciplinary research culture and Biomedical Engineering}

Biomedical engineering is inherently interdisciplinary, it prides self by pulling new perspectives from research fields such as electronics, programming, humanities, culture and psychology, and as such, invite specialists from alternative disciplines to validate insights outside of traditional practices. For affective computing, the study of human psychology is vital for the detection and regulation of emotions using technology. The term lends itself towards a vast selection of applications and research topics, to an extent that is impressive without doubt, but in the expense of obscurity when trying to determine a meaning that is inclusive. For example, biomedical engineering may be rightfully allocated to the production of prosthetic intended for physical rehabilitation, while on the other hand, it serves as a relevant label for say, a biofeedback system design for guided meditation. What binds these two is the appropriation of biomedical technology and data, which can be divided into two major essential categories: physiological sensors and actuators.

\textbf{Sensors}: These are the components that are responsible for acquiring electrical signals from the body, purposed to measure specific biological, chemical, or physical processes that occur. Section X.X provides a general overview of the common types of sensors modalities as well as their affordances.

\textbf{Actuators}: This describes some form of output mechanism that is being manipulated by the sensor data. The process in which we perceive this representation of physiological activity describes the foundation of biofeedback. In some design contexts, such actuation systems may be referred to as interactive artefacts.

As we begin to appreciate the value of such collaborations outside of the strict by medical domain, we will present our research efforts to continuously defend the inclusion of a aesthetics, primarily in the scope of emotional modelling technology, but also in the broader scope of biomedical science. We foresee a co-benefit between the appropriation of physiologically-centred systems for artistic practices, enhancing our sensory engagement with technology. And then, taking into consideration the novel aesthetic experiences to reconstruct our comprehension of live biodata as a resource for health and well-being, 

\section{Social Signal Processing}

The field of Social Signal Processing (SSP) can be tied to the increased importance of emotional intelligence in Human-Computer Interaction design \cite{cristescu_emotions_2008}. Social Signal Processing revolves around the monitoring of non-verbal behaviour to analyse social interactions. The attention directed towards non-verbal communication can be justified as a method of extracting social signals that's hard-wired in the human brain \cite{vinciarelli_social_2009-1}. 

\subsection{Non-Verbal Communication 2}

A fundamental objective of this work is to explore methods of communicating emotional or affective states without a dependency on spoken language. Non-verbal communication is a term that encompasses a variety of modalities such as posture, physical gestures and facial expressions to convey emotions, feelings, and messages beyond the use of words \cite{knapp_nonverbal_2009, richmond_nonverbal_2011}. This can be interpreted to augment meaning alongside the verbal channel during interaction, or by itself in circumstances where there are only non-verbal channels present. 

Non-verbal signals can be described as communicative or informative. A signal is produced consciously in effort to convey a specific meaning is communicative. On the other hand, when the user emits signals unconsciously, without an intended meaning, it is informative \cite{vinciarelli_towards_2011}. The unconscious quality of particular informative non-verbal behavioural cues grants a level of authenticity, advocating the label of \textit{honest signals} by Alex Pentland \cite{pentland_honest_2010}.

\subsection{Embodied Emotions}

The association between emotions and bodily expression in human and animals was first described by Darwin \cite{darwin_expression_2013}, which has been followed by numerous studies in social psychology, human development (and more recently HCI \cite{alaoui_movement_2012, gillies_creating_2018, fdili_alaoui_strategies_2015}) to address the communication of emotions from the human body \cite{gunes_lab_2008}. In this work, we will be working with the existing non-verbal modalities of gesture and posture, which are considered aspects of kinesics. They are both executed from the body and have the capacity to transmit social messages, but can be differentiated by their degree of intentionally and kinematic quality. Gestures are often (though not always) associated with movement and classed as communicative, as they are performed consciously \cite{vinciarelli_towards_2011}. A subcategory of gestures, known as \textit{adaptors} are performed unconsciously which may indicate changes in arousal or anxiety \cite{hans_kinesics_2015, neff_dont_2011}. 

Social signals addressed from postures are commonly a result of unconscious behaviour making these amongst the most honest and reliable non-verbal cues according to Richmond and McCroskey \cite{richmond_nonverbal_2011}. In a seminal work on posture and communication, Scheflen proposes three main social messages to be extracted from an interaction \cite{scheflen_significance_1964}, to characterise the posture as inclusive or non-inclusive and to assess the the level of engagement and rapport \cite{vinciarelli_social_2009}. Rapport can be associated with postural mimicry (or mirroring), which has been linked to smoother interactions and greater empathic understanding \cite{chartrand_chameleon_1999}. 

\section {Measuring Aesthetics}

The enquiry into defining aesthetics and digital representation could rightfully be reserved for an entire thesis in of itself, in which circumstance would there remain indeterminate concepts beyond its conclusion. To overcome an ever-expanding topic, we will begin by establishing a firm aesthetic criteria based upon two perceptual qualities:

\begin{itemize}
    \item \textbf{Spatial:} The organisation of distinct elements based on their relative position to one another.
    \item \textbf{Rhythmic:} The dynamic presence for which perception occurs over the time domain.
\end{itemize}

This feature space aims to be generalisable towards any kinds of mediums and modalities. 

\section{Somaesthetic Design Principles}

% Proceeding from the key term aesthetics
If we concur with the baseline understanding of aesthetics that was defined in the title constructs, we can begin to formulate a concise understanding of this phenomenon when we recognise that aesthetic perception is not restricted to our 5 primary sense organs. Shusterman's philosophy expresses the importance of our entire body, and how this is has been used not just to perceive, but how one interacts with their surroundings. Through Shusterman’s (Somaesthetic) theory, we can assure that aesthetic experiences are not strictly bounded to a gallery setting or a formal arts education, since the body is inherently capable of cultivating aesthetic value in everyday life. This ability is enhanced through performative practices, such as dance \cite{eric_c_mullis_performative_2006}

Somatic practitioners are known to intentionally disrupt habitual actions in order to bring out a third-person perspective, that is to comprehend their experiences from the outside-in, demonstrated in \ref{}, for example. This process leads us to understand the pinnacle function of technological intervention in contemporary Soma Design theory.

% Other (more interesting) research communities share these ideas also
In the wider scope of body-centric HCI research, we include Somaesthetic design as an individual component amongst literature from other research communities, broadly speaking, studies of Moving Computing [MOCO], Tanglable Interfaces [TEI] and Social Robotics, to nominate some key examples. Soma and Somaesthetic design researchers have sustained a profound voice in pushing Affective Computing away it's traditional frameworks of emotion detection, but instead considering emotional experiences enriched by computer mediated interaction (see Section \ref{affective_computng_lit_review}, The Interactional Approach). A plethora of articles and case-studies can be found in the bibliography, with records dating back as far as 200X that demonstrate and assert the relevance of this methodology.

\textit{Somatic connoisseurship} is granted to those that hold a comprehensive training and experience in one or more relevant body-centred practices, which certifies the appropriate facilitation of such activities to somatic laypersons. 

By navigating such theoretical progressions in alternative research fields, we can begin recognise patterns by which the adoption of aesthetics makes way for more emotionally relevant systems.

\section{Interactional Design Perspective}

- Traditional affective computing applications - informational View

-An  interactional  perspective  on  design  will  not aim to detect a singular account of the “right” or “true” emotion  of  the  user  and  tell  them  about  it,  but  rather  make  emotional  experiences  available  for  reflection

-When there is a continuous dialouge, we introduce affective loop experiences

These concepts align with non-representational views of cognition, as it does not rely on a reductionist model of emotion.  

% \section{Conclusion}

