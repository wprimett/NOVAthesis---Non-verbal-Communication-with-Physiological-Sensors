%!TEX root = ../template.tex
%%%%%%%%%%%%%%%%%%%%%%%%%%%%%%%%%%%%%%%%%%%%%%%%%%%%%%%%%%%%%%%%%%%%
%% chapter3.tex
%% NOVA thesis document file
%%
%% Chapter with a short latex tutorial and examples
%%%%%%%%%%%%%%%%%%%%%%%%%%%%%%%%%%%%%%%%%%%%%%%%%%%%%%%%%%%%%%%%%%%%

\typeout{NT FILE chapter3.tex}

%!TEX root = ../template.tex
%%%%%%%%%%%%%%%%%%%%%%%%%%%%%%%%%%%%%%%%%%%%%%%%%%%%%%%%%%%%%%%%%%%%
%% chapter3.tex
%% NOVA thesis document file
%%
%% Chapter with a short laext tutorial and examples
%%%%%%%%%%%%%%%%%%%%%%%%%%%%%%%%%%%%%%%%%%%%%%%%%%%%%%%%%%%%%%%%%%%%
\chapter{Literature Review}
\label{cha:lit_review}

In this chapter, we explore the current state of the art and examine existing research efforts related to non-verbal communication and physiological data. We divide the research topics into the following areas, namely: the social impact of sharing physiological activity in human-human interactions, the appropriation of biosignals in creative practice and performance contexts, and methods for inferring emotional states from physiological data. After presenting a comprehensive collection of studies, we articulate a personal perspective, justifying the intersection of these topics and initiating the design principles that materialise in the research outcomes described throughout this thesis.

\section{Communication and Connection}
\label{lit_review:biosignals_sharing}

In the pursuit of communication with embodied sensors, we can direct our interest to the topic of sharing biosignals or \textit{foreign live biofeedback} to support social interactions, where users are provided real-time feedback on another person's physiological state, as it is described in Ewa Lux et al.'s \cite{lux_live_2018} review article, listing 20 use-cases under this category. This includes a number of studies on sharing physiological signals (predominately heartbeats) between users to enhance remote presence. Such feelings of interpersonal connectedness can be accredited to a kind of anatomical transparency. Beyond that, however, it's not well understood on an emotional level the extent to which this information can be interpreted by a foreign body.

In \citeauthor{slovak_understanding_2012}'s study into sense-making with interpersonal biofeedback for heart rate, possible interpretations are classed as HR as information and HR as connection \cite{slovak_understanding_2012}. The article compares visual and aural systems for this purpose, showing that auditory feedback of the cardiovascular activity can support HR as connection in a remote setting where greater physical distances actually improved the feeling of connectedness, by which \textit{``participants suggested that less context was better''}. With HR as information, on the other hand, \textit{``participants consistently said that there was a need for context in order for the HR to be informative''}. Regarding this methodology, we note that a crucial dependency of the bonding experience and comfort lies within the pre-existing relationship between users, implying that this manner of HR sharing becomes less viable when situated between strangers. Similar works have considered the willingness of exposing private physiological information such as heart rate (e.g. \cite{walmink_displaying_2013}). Confronting such challenges that come with public space interventions, \citeauthor{howell_life-affirming_2019}'s study into sharing heart sounds explores the convergence between intimacy and anonymity in a way that encourages a co-present affirmation during momentary encounters \cite{howell_life-affirming_2019}. Another perspective on this approach can be seen in Lozano-Hemmer’s \textit{Pulse Park} artistic light installation, demonstrating the case for a mass shareability of pulse signals when integrated into communal structures \cite{lozano-hemmer_2007}.

Examples of shared biofeedback have been documented using other input and output modalities, such as skin conductance (EDA) and respiration being combined with light, visuals, and haptic forms of feedback \cite{frey_breeze_2018,howell_biosignals_2016,ashford_eeg_2019}. Among various configurations, these case studies are aligned to an interpersonal awareness of emotional states, without the need to define them literally. While these research outcomes bring affirmation to fostering social connectedness, we take into account use-cases for which subjects are able to convey contextual information from embodied sensor data. For example, inferring emotional traits from visual representations of another user's heart rate when supplemented with the events of a board game \cite{frey_remote_2016}. During gameplay, participants associated changes in other players' heart rates with bluffing, being upset, stress and even happiness.

\section{Aesthetics and Abstract Representation}
\label{lit_review:abstract}

In Chapter \ref{cha:technical_concepts}, we formed an understanding of the aesthetic domain in the context of the thesis research, pointing towards the acceptability of abstract representations in communication, unbounded by specified linguistic descriptors. This is to accept that if something is cultivated out of a subjects' personal life experiences, it is inherently ambiguous \cite{kaplan_esthetic_1948}. The overall objective with this is to open up to an individualistic presentation and perception of foreign live biofeedback \cite{lux_live_2018}, which in our case are driven by wearable sensors. In light of this, we take note of how abstraction is being routinely practised in the arts, playing a major role in describing established systems of contemporary art. For instance, abstract expressionism implies that personal and emotional structures are isolated and emphasised against ambiguous forms \cite{pollock_action_2008}.

\subsection*{In Modernism}

\citeauthor{kandel_two_2013} ties relevance to the concept of the beholder's share to a biological view of the human mind \cite{kandel_two_2013}, focusing on a specific period of art history, comparing two approaches presented by Alois Riegl and Sigmund Freud. Where the former is focused more on the artist's developmental psychology \cite{freud_leonardo_1999}, the article \citeauthor{kandel_two_2013} highlights the importance of aesthetic reception, suggesting that art is incomplete without the perceptual and emotional involvement of the viewer \cite{riegl_group_1999}. This follows into the notion identified as the \textit{beholder's share} (previously named the beholder's involvement), in which the perceiver unconsciously assigns their own meaning to the non-representation in accordance with their previous life experiences, stimulating an emotional dialogue. This understanding of a viewer's perception gives way to the ambiguous nature of individual expression, on the account that \textit{``when an artist produces a powerful image out of his own life experiences, the image is inherently ambiguous''} \cite{kandel_two_2013}, taking from Kris \& Kaplan's work \cite{kaplan_esthetic_1948}.

\subsection*{In Psychology}

In \cite{aviv_what_2014} the authors compare the mental processes that occur between perceiving artistic images, citing fMRI based studies that show distinctive patterns in the subject's neural activity between representational or abstract forms. More recently, this perceptual comparison has been studied to understand how abstract art can evoke feelings of psychological distance, hypothesised to help withstand situations of interpersonal and social distancing \cite{durkin_objective_2020}, drawing parallels with Riegl's theory of subjective experience. The authors develop a strong case for adopting abstract representations in supporting one's mental well-being in the long term. Essentially stating that, by exposing users to abstract representations, it is possible that they become more accustomed to sensing undefinable feelings in their everyday lives \cite{durkin_objective_2020}.

% In parallel, we are taking insight from the non-classical view of emotion derived from Lisa Feldman Barrett's Research. Both of these concepts highlight the role of the perceiver to induce meaningful inference.

% was further explored by art historians Ernst Kris and E.H. Gombrich,
\subsection*{In Design}

The expressive utility of abstract representation has been taken fondly by interaction design practices, notably rooted in the seminal writings of \citeauthor{gaver_ambiguity_2003} that describes finite interpretation as a limitation in data-driven digital technologies, leveraging aspects from design practice into HCI \cite{gaver_ambiguity_2003}. In a subsequent interview based on these ideas, a call for ambiguity in design is firmly articulated by Gaver, given that \textit{``we don’t need systems telling us how to live and what to do.''}, thus favouring an aesthetic appreciation that allows us to \textit{``find out own ways of leading meaningful lives''} \cite{gaver_gaffney_2007}. Ambiguity as a resource has since been embraced as part of several HCI research efforts, serving as a key contribution to a non-reductionist, non-informational view on emotionally-informed technology \cite{sanches_ambiguity_2019,howell_biosignals_2016,stahl_evocative_2014}. These principles are complemented in \citeauthor{gaver_drift_2004}'s \textit{ludic design} \cite{gaver_drift_2004}; while granting openness to interpretation, is also directed towards encouraging curiosity and exploration of an interactive artefact, lending authority to the user to curate their own experience of it. This later made its way into case studies advocating for creative participation, in a variety of contexts, including that of architecture and musical instrument design \cite{harriss_ludic_2010,mcpherson_designing_2016}.

\subsection*{Therapeutic Qualities in Artistic Creation}

As we consider the function of aesthetics in representing one's physiological activity, we are able to proceed with our underlying research goal of supporting non-verbal communication strategies. Highlight the relevance of non-verbal communication in therapeutic settings, so in addition to performance contexts, we are also interested in arts-therapy practices, which also rely on aesthetic engagement.

In the following article entitled, \textit{``The Aesthetic Turn in Mental Health''}, \citeauthor{samaritter_aesthetic_2018} assess the potential benefits music therapy and dance movement therapy can have on one's mental well-being, recognising the sensory-expressive experience of the maker, as opposed to an audience \cite{samaritter_aesthetic_2018}. Five core themes are derived from a literature review and comments taken from experts in the field, which are briefly as follows: requiring embodied presence, to apply the senses kinaesthetically and musically; somatic resources, supporting anatomical, visceral and neuropsychological functioning; articulation and expression of emotional content; sensitivity to non-verbal communication, and enactive empathy; The non-verbal or pre-verbal character of aesthetics to support diversity and cross-cultural interaction.

Amongst these themes, considered harmonious with our research goals, we can interpret that in these scenarios, that collective sense-making can be considered part of the experience. Here, we can make a connection between \citeauthor{gaver_drift_2004}'s' Ludic Design principles \cite{gaver_drift_2004} with the function of \textit{``enactive engagement''}, encouraging participants to experiment freely with new possibilities, cultivating a bond with the system and one another. While it is not explicitly mentioned here, we may suggest that a creative provocation can help foster the intergroup relationships which participants have not already acquainted i.e strangers amongst each other. Related reports have observed children exercising non-verbal forms of communication to maintain group dynamics during dance-movement therapy \cite{ylonen_kinaesthetic_2009}, engaging with foreign communities through common dances, experiencing shared embodiment \cite{hoppu_other_2013}, staying responsive to partnering movements through kinaesthetic listening \cite{ylonen_bodily_2003}, and how a non-judgemental attitude helps participants to share stories with one other during arts-therapy sessions \cite{kalmanowitz_out_2016}.

The article also establishes the essential role of an experienced somatic practitioner that is given the responsibility of guiding these sessions in a safe and constructive manner. \textit{Somatic connoisseurship} is granted to those that hold a comprehensive training and experience in one or more relevant body-centred practices, which certifies the appropriate facilitation of such activities to somatic laypersons \cite{samaritter_aesthetic_2018}.

% With an alternative approach in mind, we can start to move away from goal-oriented emotion detection tasks and instead, prepare experiences that subtly adapt the user's cognitive behaviours over time.

% \begin{figure}[htbp]
% 	\centering
% 	\includegraphics[width=0.8\textwidth]{Chapters/Figures/Concept_Venn.png}
% 	\caption{Intersection of Topics.}
% 	\label{fig:Concept_Venn}
% \end{figure}

 % Artworks and How Do We Experience Them?}: \url{ https://link.springer.com/content/pdf/10.1007%2F978-3-319-14090-2.pdf}
% !!!
\section{Wearable Technology in Creative Practice}% !!!
\label{lit_review:biosignals_creativity}

The integration of wearable devices in creative practice and artistic performance, in particular, provides motivation to present physiological activity in provocative ways for which an audience can empathise with the performer while they are being exposed to their inner state, as explained by \citeauthor{francoise_designing_2017} \cite{francoise_designing_2017}. Such performer-audience relationships can be seen as a collaborative process on account of the \textit{beholder's involvement}, asserting that the role of the perceiver is to create meaning \cite{kandel_age_2012}. In this section, we review some examples that achieve this by utilising the aesthetic value that's extracted from wearable sensor data to convey a sort of emotional meaning.

% An example of presenting bodily data as performative artefact is of physiological sensors by musicians to generate sound.
Within the broad domain of wearable technology for performance environments, Laetitia Sonami stands as a key pioneer in building and performing with new digital instruments since 1991, presenting the \textit{Lady's Glove}, capable of sensing intricate properties of the hand's shape and motion (as described in \cite{Bongers2000PhysicalII}). Looking toward other wearable sensing strategies, \citeauthor{aly_appropriating_2021} conducted a review of biosensor modalities in the scope of designing biosignal-driven instruments, which explains the affordances of EMG signals to depict significant motor actions that are revealed through internal muscular functions, praising high degrees of expressiveness \cite{aly_appropriating_2021}, as represented amongst corresponding artistic actions and case-studies (e.g. \cite{francoise_coda_2022,erdem_vrengt_2020,lesaffre_sonic_2017,caramiaux_understanding_2015}). The cited studies (\cite{erdem_vrengt_2020}, in particular) discuss the use of non-visible aspects of motion observed as internal anatomical mechanisms, performed through one's learned intuition over deliberate consciousness. In a similar vein, \citeauthor{tanaka_intention_2015} describes the EMG signal as the performer's intention to make a gesture before it's percieved \cite{tanaka_intention_2015}. Much of this work is recognised under the NIME research community (see \cite{fiebrink_rebecca_reflections_2020}, as an example), where the given systems can be compared to traditional musical instruments, granting virtuous control through attentive practice developed over time \cite{wu_supporting_2017}. Knowing this, we also point to works that incorporate wearable sensors alongside traditional forms of musical performance, such as Terminalbeach's \textit{Heart Chamber Orchestra} \cite{votava_heart_2012} or \citeauthor{lyon_biomuse_2010}'s \textit{Biomuse Trio} \cite{lyon_biomuse_2010}.

A recent review by \citeauthor{giomi_somatic_2020}, provides insights into somatic sonification approaches. While continuing to work with sound-based interactions, these are more suitable for dance performances, whereby the user is performing with their body on stage, and the system is used to better communicate their experience, though motor actions can also be influenced by sound. “Écouter/Toucher” details non-conscious and involuntary motor responses to pairing movements while muscle activity is mapped to sound feedback \cite{giomi_listening_2018}. As part of \citeauthor{latulipe2010exploring}'s “Bodies/Antibodies” performance, for example, wearable accelerometers are used to interactively control ‘secret representations’ of abstract forms, alongside pre-made animations \cite{latulipe2010exploring}.

Different types of inputs can be used to obtain the respective dance data sets. Alaoui and colleagues \cite{fdili_alaoui_seeing_2017} distinguish among: (i) positional data retrieved by employing motion capture systems; (ii) movement dynamics (such as acceleration/deceleration), recorded by means of inertial sensors; and (iii) physiological information, obtained from biosignal sensor technologies. Correspondingly, authors created a methodology that combines multimodal capture with recognition of Laban Effort qualities \cite{fdili_alaoui_seeing_2017}; this is used to assign mappings between gestures and sonic parameters, directing attention away from task-oriented interactions. Similarly, this has also been done to develop visualisation systems that are informative of performed movement qualities \cite{alaoui_interactive_2015,hussain_evaluating_2019,jego_workflow_2019}. Rostami et al. \cite{rostami_bio-sensed_2017} created five design concepts for interactive performance adopting bio-sensing and bodily tracking technologies.
% from chapter 6.2.2
% Zhou et al. have recently conducted a systematic review of the past twenty years of dance literature in HCI \cite{zhou_dance_2021}. They identified four main categories of technological approaches: Physiological Sensing; Multisensory Perception; Movement Quality; and Agent Collaboration.
% They distinguished among different approaches for collecting information from the body: Positional data retrieved by motion capture; Movement dynamics recorded by inertial sensors; and Physiological information obtained from biosignal sensors.
% Aly et al. conducted a review of biosensor modalities in the scope of HCI, which explains the affordances of EMG signals to depict significant motor actions that are revealed through internal muscular functions. \cite{aly_appropriating_2021}.

Given that it can be unintuitive to hard-code the detection and analysis of performative gestures, machine learning can be utilised to allow the system to associate features of the sensor data with gestures after being trained on a set of examples. This approach has shown promising results regarding the exposure of expressive traits within the subject's movement \cite{caramiaux_understanding_2015}. This echoes early research achievements such as Wekinator \cite{fiebrink_rebecca_reflections_2020} and puts forth recent progress such as that exhibited by the Teachable Machine \cite{carney_teachable_2020}. Interactive machine learning \cite{fails_interactive_2003}. By means of tools like Wekinator or Teachable Machine-like paradigms, it is possible to define actions, visual material, movements or gestures according to example data curated by the responsible user.
% The following state-of-the-art review points to a number of use-cases that incorporate contemporary machine learning methods.
To summarise these motives, we can assume a mutual view by which the output of the machine learning system is considered to play a role in communication, which can be achieved through audiovisual augmentation that compliments the movement \cite{francoise_coda_2022}, or even more literally in the form of an artificial agent that performs on stage \cite{liapis_learning_2018}.

\section{Inferring Emotional States from Physiological Data}
\label{lit_review:affective_computing}

% In this section, we will summarise the common practices found in Affective Computing research over the past two decades and how this has evolved to the current state-of-the-art. The methods discussed in the following section are used as a reference to the typical assumptions adopted in the field, as for the rest of the thesis, we will be receptive to alternative views for designing emotionally informed systems, leveraging principles embraced by \citeauthor{hook_interactional_2008} \cite{hook_interactional_2008}, \citeauthor{barrett_functionalism_2017} \cite{barrett_functionalism_2017} and others cited throughout.
\subsection*{Background}

 Given an intriguing appeal for developing emotionally-informed relationships between humans and machines, the attention here is directed specifically to the physiological constructs of emotion and the appropriation of wearable sensors. The theoretical validation of which is partly derivative of the theories of emotions put forth by William James and Carl Lange, claiming that emotions are the result of bodily changes triggered by external stimuli \cite{james_principles_1890,cannon_james-lange_1927}. Many have come out to challenge these ideas, with more advanced perspectives openly being addressed in Picard's original report, \textit{Affective Computing} \cite{picard_affective_1995}, complimenting the inherent complexities and open debate around the subject of emotional technologies. In this section, we will summarise the common practices found in Affective Computing research over the past two decades and how this has evolved to the current state-of-the-art.

Generally speaking, studies in this field are expected to measure the emotional responses that are elicited whilst perceiving an overt stimuli, comprised of predetermined emotional triggers \cite{bota_review_2019}. For this purpose, there are already several datasets of scientifically validated elicitation material organised according to a set of affective categories (e.g. \cite{yang_affective_2018,koelstra_deap_2012,bradley_affective_2007}); a collection of relevant studies can be found in a recent survey \cite{sanches_hci_2019}. Among these, the informational outlook adopted in some of these examples has since been contrasted with an interactional approach that's used to emphasise the social aspects of emotions. We may assume its origin from \citeauthor{boehner_how_2007}'s enquiry into affective measurement short after the field came into fruition, calling for technology that support \textit{``people to understand and experience their own emotions''} \cite{boehner_how_2007,boehner_affect_2005}. We find the interactional approach work in compliment with the non-classical view of emotion derived from Lisa Feldman Barrett's writings \cite{barrett_how_2017}, highlighting the role of the perceiver to induce meaningful inference, moving away from discrete categorisation models of emotion. Or, as \citeauthor{hook_interactional_2008} puts it, \textit{``An interactional perspective on design will not aim to detect a singular account of the “right” or “true” emotion of the user and tell them about it as in a prototypical affective computing application, but rather make emotional experiences available for reflection . Such a system creates a representation that incorporates people’s everyday experiences that they can reflect on. Users’ own, richer interpretation guarantees that it will be a more “true” account of what they are experiencing.''} \cite{hook_interactional_2008}.

\subsection*{Emotional Interaction}

The interactional approach asserts that though emotion has biophysical aspects, it is dynamically experienced and appreciates \textit{``emotion as a social and cultural product experienced through our interactions''} \cite{boehner_how_2007}. Here is room for an affective computing ideal that does not look at the body as \textit{“an instrument or object for the mind, passively receiving sign and signals, but not actively being part of producing them”}, as phrased by Höök when referring to dominant paradigms in commercial wearable sensor applications \cite{hook_kristina_affective_2012}. However, how to best address non-verbal behaviour beyond measuring cues and signals is unclear. Most studies in affective computing revolve around affect recognition from emotion detection and bodily data classification \cite{bota_review_2019}. Research in this field commonly revolves around the detection of four basic emotions: fear, anger, sadness, and joy \cite{picard_mit_1995}. Such alternatives can alleviate the function of mapping discrete emotions into a statistical model bound to universal interpretation, but rather let humans make sense of each other's expressions in subjective ways, ambiguous and contextually adaptive. Recent adoptions of an interactional approach have been exemplified in the following studies \cite{sanches_ambiguity_2019,umair_thermopixels_2020,fosh_see_2013}, in addition to those discussed previously in Section \ref{lit_review:biosignals_sharing}.
% Such reports open themselves to practices and discussions around using abstract representations. \citeauthor{hook_kristina_affective_2012}

While the use of structured elicitation material allows for statistical evaluations and successfully determines a numerical accuracy of emotional inference \cite{bota_review_2019}, interactional studies are implicitly more dependent on qualitative analysis (e.g. \cite{hook_embracing_2018,howell_life-affirming_2019}). We may suggest also that in lack of stimuli, an interactional approach relies upon its users and surroundings to stimulate expressive dialogue \cite{gonzalez_dance-inspired_2012}. From this, we'd like to express how shifting from the informational view provides an appropriate base when taking sensory interactions out of controlled lab environments, hence being more adaptive to the unpredictable nuances that present themselves in everyday life \cite{brown_into_2011,stjerna_aspects_2013,maki-petaja_aesthetic_2014}.
% This is something we consider throughout the thesis research, discussed in further detail in the following sections.

% Picard's critics describe her concept of emotion as ``objective, internal, private, and mechanistic''. They say it reduces emotion to a discrete psychological signal occurring inside the body that can be measured and which is an input to cognition, undercutting the complexity of emotional experience.

To summarise, we consider that the interactional view supports the theoretical grounds for communication strategies with wearable sensors by welcoming aesthetic representation and henceforth, ambiguity. This leads us to broaden the proposed affective loop framework proposed in \cite{hook_affective_2009}, by engaging multiple subjects simultaneously, inclusive of third-person perspectives, supporting empathy, communication, and joint acts of perception to transpire \cite{turmo_vidal_designing_2021,francoise_designing_2017}.

\subsection*{Measuring Affect in Social Contexts}
\label{lit_reivew:ssp}
% Social siganls
In face-to-face interactions, subjects make use of \textit{``Body Language''} as a form of non-verbal communication, assuming actions such as gaze, gestures, posture and navigation of interpersonal space \cite{dobre_immersive_2022}. Psychology literature admits to the limited ability we have as humans to interpret non-verbal behaviours during social interactions \cite{joseph_emotional_2010}. However, by recognising the coordination with emotions and physiological changes \cite{mayer_human_2008}, we can consider sensory interventions as a means to help convey and interpret social cues emitted from other people.

In the interest of bridging two prominent topics that are aligned with our research goals, we give attention to an article by Chanel and Mühl \cite{chanel_connecting_2015} which surveys the use of physiological data to facilitate affective non-verbal communication, combining concepts from Affective Computing and Social Signal Processing. The authors acknowledge the case for traditional emotion recognition tasks and explain that physiological activity can also be related to several social processes, such as empathy \cite{levenson_empathy_1992}. It's also explained how activity from the central nervous system correlates to affective and cognitive states during social interaction, where social emotions include shame, embarrassment, gratitude and admiration. These physiological changes are considered for the most part involuntary and therefore hold little to no bias.

Despite knowing that there's a strong correlation between physiological data with affect, authors identify a lack of studies directed towards social interaction, and the most studied non-verbal modalities are in fact facial expressions and tone of voice. This can be partially justified given that in conventional social situations, these internal signals aren't so easily perceived \cite{vinciarelli_social_2009}. In this persuasion, two research directions are proposed, the first is to display social cues to foreign users through physiological data, and the second involves analysing physiological activity among multiple users to facilitate collective interaction. The latter approach compliments \citeauthor{slovak_understanding_2012}'s proposal for \textit{composite signals} for HR as information (i.e., combining physiological signals from different people together) \cite{slovak_understanding_2012}.

\section{The Embodiment of Expression, Self-Report Strategies}
\label{lit_review:self_report}
% \textcolor{red}{another title?}

Self-reports are often considered a necessary resource in studies of psychology and mental well-being, suitably transferable to Affective Computing also. Generally speaking, this serves as an assessment of the subject's cognition, emotion, motivation, behaviour, or physical state, accepted as a personal judgement. Such reports can emerge in many different forms; questionnaires and interview transcripts are commonplace, but not inclusive \cite{barker_self-report_2016}. Taking the understanding of embodied emotions described in Section \ref{background:ebodiment_emotions}, we highlight existing methods of reporting that comply with non-verbal modes of expression, specifically those linked to bodily attributes.

\subsection*{The Self-Assessment Manikin}

The Self-Assessment Manikin (SAM) has proven to be a successful method of mapping emotional responses to a set of pictorial representations, pre-arranged to visualise 5-point scales of valence, arousal and dominance, thus considered highly compatible within the scope of Affective Computing research \cite{broekens_affectbutton_2013,zeigler-hill_self-assessment_2017}. The method was introduced as a way of overcoming the semantic dependencies in traditional self-reporting, which carry major limitations related to the tedious data organisation required and being poorly adaptable between different languages \cite{zeigler-hill_self-assessment_2017}. However, where the SAM provides an effective way of capturing emotional states related to a specific object or event, there remains lacking feasibility for representing traits relating to one´s habitual patterns of emotion.

\subsection*{Body Maps}
% \textcolor{red}{too long? Can it be split somehow?}

The term ‘body mapping’ is defined as the process of creating human body images \textit{``to visually represent aspects of people's lives, their bodies and the world they live in''} \cite{gastaldo_body-map_2012}. This term has been used for health contexts (e.g. HIV/AIDS) in the last two decades, as well as in the context of occupational health and safety for almost 50 years \cite{gastaldo_body-map_2012}. \citeauthor{de_jager_embodied_2016} conducted a systematic review of body mapping, identifying several attributes, such as social justice, community development tool, planning and psychological assessment \cite{de_jager_embodied_2016}. The body maps Gastaldo et al. \cite{gastaldo_body-map_2012} presented in their research are free-form and unconstrained; there is no predefined structure (such as a preexisting body outline) and include contextual elements such as annotations. A more constrained body map, with an outline of the human body, is used by Windlin et al. \cite{windlin_soma_2019}, to reflect on experiences through drawings and notes: \textit{“these maps became a rich source of data for first-person perspectives of bodily experiences and sensations”}. For the body maps, the authors adapted an earlier body outline from Loke et al. \cite{loke_bodily_2012}. A similar body map has been used in related soma design research, for example by Höök et al. \cite{hook_soma_2019} and Tsaknaki et al. \cite{tsaknaki_teaching_2019}.
% Núñez-Pacheco and Loke introduced the concept of felt-sensing -- inner exploration of bodily feeling -- using body maps \cite{nunez-pacheco_felt-sensing_2016}. Núñez-Pacheco proposed “the use of tangible materials as a way to articulate experience from the inner self”, also employing body maps \cite{nunez-pacheco_tangible_2021}.
% In a scenario closer to the artistic research herein presented, Loke and Khut \cite{loke_intimate_2014} developed an art exhibition using “interactive heart rate controlled audio-visuals with audience participation”. They used body maps as a means for audience members to describe “sensations and images they experienced within and around their body, during the interaction” \cite{loke_intimate_2014}.

We would like to suggest here, that these representations become more difficult to interpret from one person to another as they are used to document more abstract sensations in the body. When considering that, for instance, two persons may be highlighting the same areas, producing illustrations that look very much alike while trying to describe an experience that is entirely different from one another (and visa versa). Unlike the SAM, there is no definitive emotional inference to be made from an individual bodily drawing, this is formed by a process of interpretation when presented alongside supplementary information. Body maps may be presented as first-person descriptions, visual annotations \cite{windlin_soma_2019}, video recordings (explored in Section \ref{preliminary_actions:modi_ws1}), or taken together with additional body maps, revealing generalisable patterns amongst them. %\cite{nummenmaa_bodily_2014}.
Nummenmaa et al. conducted a study where over 700 participants we asked to colour body map regions according to various emotional stimuli \cite{nummenmaa_bodily_2014}. By layering these visual impressions, it was possible to observe \textit{``net sensations''}, this being the collective agreement towards the bodily regions being activated when subjected to a particular emotional cue. For example, “anger” in the hands or the whole body in “happiness” \cite{davey_where_2021}. Taking inspiration from this, \citeauthor{schino_applying_2021} exemplifies how body mapping exercises enabled participants to report complex emotions that were elicited by new media artworks \cite{schino_applying_2021}.

%  \subsubsection{Limitation of Body Maps in HCI}
 Recent literature on soma design has aimed to overcome the temporal limitations of body maps, as \textit{``they exist as a snapshot or state representation''} \cite{tennent_articulating_2021}. To solve this limitation, Tennent et al. propose the concept of \textit{``soma trajectories''}, i.e \textit{``how a user feels through an interaction, both in body and mind''} \cite{tennent_articulating_2021}. In Sections \ref{case_studies:latent_steps} and \ref{case_studies:modi_dis}, we further detail what we perceive as major limitations of body mapping strategies and investigate new ways of overcoming them.

\subsection*{Tangible Artefacts}

The visual self-reports described above are assumed to be viewed on paper or digital display. As an alternative, we can make a note of efforts into physical self-reporting mediums. Núñez-Pacheco proposed \textit{“the use of tangible materials as a way to articulate experience from the inner self”}, employing the structure of body maps \cite{nunez-pacheco_tangible_2021}. Further, \citeauthor{isbister_sensual_2007} examines the tactile visual qualities of physical objects that can be mapped to emotional imagery, with an interest to overcome the inter-cultural limitations of measuring affect \cite{isbister_sensual_2007}. More recently, developments into tangible mechanisms for logging affect, specifically designed for older adults have shown their potential to support emotional reflection and well-being \cite{gooch_designing_2022}.

\subsection*{Physiological Responses}
%  Sensor data as a report
A recent survey describes how affective physiological measures alone can subside the use of traditional self-reporting procedures \cite{barker_self-report_2016}. However, much of this work is geared towards consumer appeal and advertising, settling for basic uni-dimensional forms of emotional measure. \citeauthor{ciuk_measuring_2015} explicitly opens up such limitations given the context of gauging political attitudes, where physiological measures fall short in comparison to self-reports when accounting for emotional nuances \cite{ciuk_measuring_2015}. To our knowledge, studies that combine physiological data with visual self-reporting are hard to come by. One example that successfully correlates these two forms of assessment was carried out by \citeauthor{jung_role_2017} seeks to understand how emotional states can be influenced by one's awareness to their internal bodily states (e.g. heart rate) \cite{jung_role_2017}.

Concluding from all of this, we are far more interested in adopting a multimodal approach where sensor data is rather used to organise and navigate between user-reported states, not acting as the inference itself.

%  critisims of William James theories of emotion, that does not consider differences in anatomical sensitivity between individuals

\section{Summary}
\label{lit_review:conclusion}

In this work, we aim to develop and evaluate methods of sharing physiological activity, in an approach that is abstracted from raw signals and without explicit associations to linguistic descriptors. Through visual, auditory and tactile feedback, we intend to generate new representations of the subject's inner state inferred from analysing physiological signals, which can then be transmitted to foreign users as a means of communication. In reflection of the studies referenced above, we embrace non-representations and aesthetics that can engage the brain in new ways, developing new cognitive and emotional associations.

In Figure \ref{fig:Abstracted_Representations}, we show a simplified plot of how physiological activity can be presented either before or after inference. The leftmost end of the spectrum leans towards the ideas discussed in Section \ref{lit_review:biosignals_sharing}, where live biofeedback depicts the low-level features of the signal. The rightmost end of the spectrum describes what is achieved in cognitive-based emotion recognition systems, where the data is computationally analysed to produce emotional inferences for the user, manifested as high-level descriptors (e.g. fear, joy, surprise, etc...). We intend to demonstrate a middle-ground, for which it's not necessary to determine a particular socio-affective inference, nor intended to display. To achieve this, we take inspiration from performative uses of embodied sensor data, introduced in Section \ref{lit_review:biosignals_creativity}, enabling features of the signal to generate abstract representation, to be perceived by a foreign body.
% In this case, it's expected for the external user to make socio-affective inferences from the abstracted representation.

\begin{figure}[htbp]
	\centering
	\includegraphics[width=0.7\textwidth]{Chapters/Figures/Abstracted_Representations.png}
	\caption{Types of outputs for representing physiological activity.}
	\label{fig:Abstracted_Representations}
\end{figure}

Extending previous work surveyed in this section, herein we delve into the affordances of contemporary machine learning techniques as a means of generating new representations of biosensor data. We consider this initiative to be a novel contribution to the field given the impression that machine learning systems are commonly purposed to derive high-level emotional descriptions for emotion recognition tasks. To stay persistent with the thesis objectives, these generative representations should align with the interactional perspective in regards to the non-reductionist principles summarized in Section \ref{lit_review:abstract}. The aim is not to detect a singular ``right'' or ``true'' emotion, but rather, to inspire expressive dialogue and emotional reflection \cite{hook_affective_2009}. Our incentive for adopting non-representational machine learning solutions in this context is to exhaust the output possibilities when mapping decoded sensor data of relatively low dimensionality and embracing high-granularity results that can expose expressive nuances that may not be so salient in the raw data alone.
% burrowing something pre-exsitsing to describe something conceptual \cite{lakoff_metaphors_2003}
% In addition, we would like to study upon the idea that machine learning systems can improve the level of personalization in affective interactive systems compared to what is normally expected.

In summary, we offer the following design principles that will be adopted into practical assessment: (i) Communication mediums should not infer emotional states that hold a predetermined meaning; (ii) Physiological sensor data elicits feedback, however, the content of which should not directly mirror the raw biological functions that correspond; (iii) Users are granted the responsibility to explore the affordances of a system according to their bodies and interpret the representations that are produced in that process; (iv) Sensor data alone is not a sufficient means of expression, rather considered a way of navigating the aesthetic functions informed by the user and their surroundings.

% Due to the ambiguity of the feedback, the authors found that the device often led to conversations about the wearer’s feelings. Generally, when providing ambient or wearable live biofeedback for social interaction, one  needs to consider users’ willingness to share their private physiological information

% Given the ambient nature of these interventions, designers should consider how one is exposing their private physiological information
