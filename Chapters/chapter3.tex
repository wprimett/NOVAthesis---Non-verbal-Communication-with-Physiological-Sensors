%!TEX root = ../template.tex
%%%%%%%%%%%%%%%%%%%%%%%%%%%%%%%%%%%%%%%%%%%%%%%%%%%%%%%%%%%%%%%%%%%%
%% chapter3.tex
%% NOVA thesis document file
%%
%% Chapter with a short latex tutorial and examples
%%%%%%%%%%%%%%%%%%%%%%%%%%%%%%%%%%%%%%%%%%%%%%%%%%%%%%%%%%%%%%%%%%%%

\typeout{NT FILE chapter3.tex}

%!TEX root = ../template.tex
%%%%%%%%%%%%%%%%%%%%%%%%%%%%%%%%%%%%%%%%%%%%%%%%%%%%%%%%%%%%%%%%%%%%
%% chapter3.tex
%% NOVA thesis document file
%%
%% Chapter with a short laext tutorial and examples
%%%%%%%%%%%%%%%%%%%%%%%%%%%%%%%%%%%%%%%%%%%%%%%%%%%%%%%%%%%%%%%%%%%%
\chapter{Literature Review}
\label{cha:lit_review}

In this chapter, we explore the current state of the art and examine existing research efforts related to Non-verbal Communication and Physiological Data. We divide the research topics into the following areas, namely: the social impact of sharing physiological activity in human-human interactions, the appropriation of biosignals in creative practice and performance contexts, and methods for inferring emotional states from physiological data. After presenting a comprehensive collection of studies, we articulate a personal perspective, justifying the intersection of these topics and initiate the design principles that materialise in the research outcomes described in Chapters \ref{cha:Preliminary_Actions} and \ref{cha:case_studies} that following.

\section{Social Signals (WIP)}
People have a limited ability to (correctly) assess other people’s mental states (i.e., mentalizing) and interpret social cues, which is vital for social behavior [Polosan et al., 2011]. [Fernández et al., 2013; Howell et al., 2016]—Studies exemplify how one can amplify social cues and/or people’s sensitivity towards such cues based on foreign-to-self signals. In this vein, studies provide live biofeedback as a means to convey social cues in order to facilitate the mentalizing process and foster social interaction. Building on this theoretical pathway, studies investigate foreign live biofeedback as a driver for social interaction (Howell et al., 2018), social connectedness (Slovák et al., 2012), social experience (Mueller et al. 2010), social engagement (Snyder et al., 2015), or social support (Walmink et al., 2013). Studies often implement live biofeedback systems for social interaction by using ambient or wearable devices. Such devices facilitate foreign and self live biofeedback at the same time because users (both self and other) can potentially perceive them. Gervais et al. (2016) found that ambient live biofeedback devices can ease social interaction, foster empathy and relaxation, and promote self-reflection (Gervais et
al., 2016). Järvelä et al. (2016) report increased heart rate synchrony for dyads at different geographical locations. Roseway et al. (2015) report that their BioCrystal system resulted in users’ more highly recognizing their physiological states and supported interpersonal communication (Roseway et al., 2015). Slovák et al. (2012) found that heart rate sharing does not improve feelings of closeness in the workplace.

With Howell et al.’s (2016) wearable foreign live biofeedback t-shirt, pairs of friends can share emotions, such as joy or embarrassment. Wearable live biofeedback systems support users to enact social performances such as emotional engagement. Due to the ambiguity of the feedback, the authors found that the device often led to conversations about the wearer’s feelings. Generally, when providing ambient or wearable live biofeedback for social interaction, one  needs to consider users’ willingness to share their private physiological information

\section{Communication and Connection with Wearable Sensors}
\label{lit_review:biosignals_sharing}

In the pursuit for communication with embodied sensors, we can direct our interested to the topic of sharing biosignals or \textit{foreign live biofeedback}, where users are provided real-time feedback on another person's physiological state, as it is described in Ewa Lux et al.'s \cite{lux_live_2018} review article, listing 20 use-cases under this category. This includes a number of studies on sharing physiological signals (predominately heartbeats) between users to enhance remote presence. Such feelings of interpersonal connectedness can be accredited to a kind of anatomical transparency. Beyond that, however, it's not well understood on an emotional level the extent at which this information can be interpreted by a foreign body.

In Slovak et al.'s study into sense-making with interpersonal biofeedback for heart rate, possible interpretations are classed as HR as information and HR as connection \cite{slovak_understanding_2012}. The article compares visual and aural systems for this purpose, showing that auditory feedback of the cardiovascular activity can support HR as connection in a remote setting where greater physical distances actually improved the feeling of connectedness, by which \textit{"participants suggested that less context was better"}. With HR as information on the other hand, \textit{"participants consistently said that there was a need for context in order for the HR to be informative"}. Regarding this methodology, we note that a crucial dependency of the bonding experience and comfort lies within the pre-existing relationship between users, implying that this manner of HR sharing becomes less viable when situated between between strangers. \citeauthor{howell_life-affirming_2019}'s study into sharing heart sounds confronts the challenges that come with public space interventions, exploring a convergence between intimacy and anonymity in a way that encourages a co-present affirmation during momentary encounters \cite{howell_life-affirming_2019}.

While these research outcomes bring affirmation into fostering social connectedness, we are also interested in cases that allows subjects to convey emotional information from embodied sensor data. This work exemplifies how users can infer emotional traits from visual representations of another user's heart rate when supplemented with the events of a board game as context \cite{frey_remote_2016}. During gameplay, participants associated changes in other player's heart rate with bluffing, being upset, stress and even happiness.

% An extensive collection of studies under this topic have been reviewed by Ewa Lux et al. \cite{lux_live_2018}. The article lists 20 use-cases that apply what's described as \textit{foreign live biofeedback}, where its users are provided real-time feedback on another person's physiological state. For each use-case, the authors state the main focus, that being user experience, social interaction or stress management, the physiological modality utilised and the manifestation of the feedback, classed as visual, auditory, haptic or even as game mechanics.

\section{Wearable Sensors in Creative Practice}
\label{lit_review:biosignals_creativity}

The integration of biosignals in performance provides motivation to present physiological activity in creative ways so audience members can empathise with the performer by being exposed to their inner state. In this subsection, we review some examples that achieve this by utilising the aesthetic value of physiological sensor data to convey sort of emotional meaning.

As we consider the function of aesthetics in representing one's physiological activity, we are able to proceed with our underlying research goal of supporting non-verbal communication strategies. Highlight the relevance of non-verbal communication in therapeutic settings, so in addition to performance contexts, we are also interested in arts-therapy practices, which also rely on aesthetic engagement. In the article \cite{samaritter_aesthetic_2018}, \citeauthor{samaritter_aesthetic_2018} assess the potential benefits music therapy and dance movement therapy can have on one's mental well-being, recognising the sensory-expressive experience of the maker as oppose to an audience.  Five core themes are derived from a literature review and comments taken from experts in the field, which are briefly as follows:

\begin{itemize}
% \item Arts support and require embodied presence
\item Arts-therapy appeal to somatic resources, supporting anatomical, visceral and neuropsychological functioning
\item Arts support articulation and expression of emotional content.
\item Arts support sensitivity to non-verbal communication between participants. Furthermore, enactive empathy can be considered a core aspect arts-therapy practices.
\item The non-verbal or pre-verbal character of aesthetics was considered to support diversity and cross-cultural interaction
\end{itemize}

While experiencing a performance, and the audience is engaged, the performer-audience relationship becomes a collaborative process, because it is that role of the perceive to make any meaning.

\section{Aesthetics and Abstract Representation}
\label{lit_review:abstract}

% A breif relevance with art history
% \begin{itemize}
% \item Going beyond the sensory enjoyment of art
% \item fMRI study comparing representational and abstract, evoking different mental patterns from the perceiver
% \item Triggering of psychological distance, relevance to emotional HCI and sharing biodata
% \end{itemize}

In the previous chapter, we form an understanding of the aesthetic domain in context of the thesis research, pointing towards the acceptability of abstract representations in communication, unbounded to specified linguistic descriptors. This is to accept that if something is cultivated out of their personal life experiences, it is inherently ambiguous \cite{kaplan_esthetic_1948}. The overall objective with this is to open up to an individualistic presentation and perception of foreign live biofeedback \cite{lux_live_2018}, that in our case, are driven by wearable sensors. In light of this, we take note of how abstraction is being routinely practised in the arts, playing a major role in describing established systems of contemporary art. For instance, abstract expressionism implies that personal and emotional structures are isolated and emphasised against ambiguous forms \cite{pollock_action_2008}. The following essay compares the mental processes that occur between perceiving artistic images, citing studies fMRI based studies that show distinctive patterns in the subject's neural activity between representational or abstract presentations \cite{aviv_what_2014}. 

More recently, this perceptual comparison has been studied to understand how abstract art can evoke feelings of psychological distance in various forms, hypothesised to help withstand situations of interpersonal and social distancing \cite{durkin_objective_2020}. Here, the authors develop a strong case for adopting abstract representations, not only to allow flexible interpretations while a subject are engaged with the system, but also with the interest of supporting their mental well-being in the long term. By exposing users to abstract representations, it's possible that they become more accustom to sensing undefinable feelings in their everyday lives \cite{durkin_objective_2020}.

% In parallel, we are taking insight from the non-classical view of emotion derived from Lisa Feldman Barrett's Research. Both of these concepts highlight the role of the perceiver to induce meaningful inference. 
% Our objective is to correlate these ideas, presenting a continuous narrative between the different subject areas and consolidate the key themes in our design principles.

In the 2013 journal article \textit{Two Modernist Approaches to Linking Art and Science}, Eric R. Kandel ties relevance to the art history concept of the beholder's share to the biological understanding of the human mind \cite{kandel_two_2013}. To articulate this methodology, Kandel focuses on a specific period within Viennise art culture, and compares two modernists ... and .... Where the former is focused more on the artist's developmental psychology, the article highlights the importance of aesthetic reception, suggesting that art is incomplete without the perceptual and emotional involvement of the viewer \cite{riegl_group_1999}. Riegal identified this notion at the \textit{beholder's share} (previously named the beholder's involvement), in which the perceiver unconsciously assigns their own meaning to the non-representation in accordance to their previous life experiences, stimulating an emotional dialogue. This understanding of a viewer's perception as essential to individual expression was further explored by art historians Ernst Kris and E.H. Gombrich, defending the idea that to produce something personal, it is inherently ambiguous [Kris, E., \& Kaplan, a. (1952). aesthetic ambiguity. in E. Kris, Psychoanalytic explorations in art (pp. 243–264). New York: international Universities Press.].


The expressive utility of abstract representation has been taken fondly by interaction design practices, notably rooted in the seminal writings of \citeauthor{gaver_ambiguity_2003} that describes finite interpretation as a limitation in data-driven digital technologies, leverging aspects from design practice into HCI \cite{gaver_ambiguity_2003}. In a subsequent interview based around these ideas, a call for ambiguity in design is firmly articulated by Gaver, given that \textit{"we don’t need systems telling us how to live and what to do."}, thus favouring an aesthetic appreciation that allows us to \textit{"find out own ways of leading meaningful lives"} \cite{gaver_gaffney_2007}. Ambiguity as a resource has since been embraced as part of several HCI research efforts, serving as a key contribution to a non-reductionist, non-informational view on emotionally-informed technology \cite{sanches_ambiguity_2019,stahl_evocative_2014}. These principles are complimented in \citeauthor{gaver_drift_2004}'s \textit{ludic design} \cite{gaver_drift_2004}, while granting openness to interpretation, is also directed towards encouraging curiosity and curiosity of an interactive artefact, lending authority to the user to curate their own experience of it. This later made its way into studies in creative partipation practices, in a variety of contexts, including that of architecture and musical instrument design \cite{harriss_ludic_2010,mcpherson_designing_2016}.

In review of the studies referenced in the previous paragraphs, we can conclude that non-representations and aesthetics can engage the brain in new ways, developing new cognitive and emotional associations. Not only to allow flexible interpretations while a subject are engaged with the system, but also with the interest of supporting their mental well-being in the long term.

By exposing users to abstract representations, it's possible that they become more accustom to sensing undefinable feelings in their everyday lives \cite{durkin_objective_2020}.
% With an alternative approach in mind, we can start to move away from goal-oriented emotion detection tasks and instead, prepare experiences that subtly adapt the user's cognitive behaviours over time.

% \begin{figure}[htbp]
% 	\centering
% 	\includegraphics[width=0.8\textwidth]{Chapters/Figures/Concept_Venn.png}
% 	\caption{Intersection of Topics}
% 	\label{fig:Concept_Venn}
% \end{figure}

 % Artworks and How Do We Experience Them?}: \url{ https://link.springer.com/content/pdf/10.1007%2F978-3-319-14090-2.pdf}

\section{Affective Computing: Inferring Emotional states from Physiological Data}
\label{affective_computng_lit_review}

% In this subsection, we will summarise the common practices in Affective Computing research over the past two decades and how this has evolved to the current state-of-the-art.

The articles discussed in the following paragraphs should be used as a reference to the typical assumptions of the field, as for the rest of the thesis, we will be receptive to alternative views for designing emotionally informed systems.
% These are explained in more detail in section \ref{lit_review:conclusion} as we begin to deconstruct some of these assumptions in order to map out our methods.

here is room for an affective computing that does not look at the body as “an instrument or object for the mind, passively receiving sign and signals, but not actively being part of producing them”—as phrased by Höök when referring to dominant paradigms in commercial sports applications [22]. However, how to best address bodily movement and engagement beyond measuring cues and signals is unclear. Most studies in affective computing revolve around affect recognition from emotion detection and bodily data classification \cite{bota_review_2019}. Research in this field commonly revolves around the detection of four basic emotions: fear, anger, sadness, and joy \cite{picard_mit_nodate}.


\begin{center}
\begin{tabular}{|p{13cm}}
\textbf{Cognitivist vs. interactional approaches}

Within the field of human-computer interaction, Rosalind Picard's cognitivist or "information model" concept of emotion has been criticized by and contrasted with the "post-cognitivist" or "interactional" pragmatist approach taken by Kirsten Boehner and others which views emotion as inherently social.

Picard's focus is human-computer interaction, and her goal for affective computing is to "give computers the ability to recognize, express, and in some cases, 'have' emotions".In contrast, the interactional approach seeks to help "people to understand and experience their own emotions" and to improve computer-mediated interpersonal communication. It does not necessarily seek to map emotion into an objective mathematical model for machine interpretation, but rather let humans make sense of each other's emotional expressions in open-ended ways that might be ambiguous, subjective, and sensitive to context.[example needed]

Picard's critics describe her concept of emotion as "objective, internal, private, and mechanistic". They say it reduces emotion to a discrete psychological signal occurring inside the body that can be measured and which is an input to cognition, undercutting the complexity of emotional experience.

The interactional approach asserts that though emotion has biophysical aspects, it is "culturally grounded, dynamically experienced, and to some degree constructed in action and interaction". Put another way, it considers "emotion as a social and cultural product experienced through our interactions".

% \begin{itemize}
% \item Boehner, Kirsten; DePaula, Rogerio; Dourish, Paul; Sengers, Phoebe (2007). "How emotion is made and measured". \item Boehner, Kirsten; DePaula, Rogerio; Dourish, Paul; Sengers, Phoebe (2005). "Affection: From Information to Interaction"
% \item Hook, Kristina; Staahl, Anna; Sundstrom, Petra; Laaksolahti, Jarmo (2008). "Interactional empowerment"
% \end{itemize}

\end{tabular}
\end{center}

% \textit{Conclude subsection justifying why interactional approach may be more suitable for interpersonal communication}
To summarise, we consider that the interactional view supports the theoretical grounds for communication strategies with wearable sensors by welcoming aesthetic representation and henceforth ambiguity. This leads us to broadening the proposed affective loop framework by engaging multiple subjects simultaneously, inclusive of third-person perspectives.

\subsection{Body Maps}

 We would like to suggest here, that these representations become more difficult to interpret from one person to another as they are used to document more abstract sensations in the body. When considering that, for instance, two persons may be highlighting the same areas, producing illustrations that look very much alike while trying to describe experience that are entirely different from one another (and visa versa).

 \subsubsection{Limitation of Body Maps in HCI}
 Recent literature on soma design has aimed to overcome the temporal limitations of body maps, as “they exist as a snapshot or state representation” \cite{tennent_articulating_2021}. To solve this limitation, Tennent et al. propose the concept of “soma trajectories”: “how a user feels through an interaction, both in body and mind” \cite{tennent_articulating_2021}.

\section{The Intersection of Affective Computing and Social Signal Processing}
\label{lit_reivew:ssp}

In interest to bridge two dominant topics aligned with our research goals, we give attention to an article by Chanel and Mühl \cite{chanel_connecting_2015} which surveys the use of physiological data to facilitate affective non-verbal communication, combining concepts from Affective Computing and Social Signal Processing.
The authors acknowledge the case for traditional emotion recognition tasks and explain that physiological activity can also be related to several social processes, such as empathy \cite{levenson_empathy_1992}. It's explained how activity from the central nervous system correlates to affective and cognitive states during social interaction, where social emotions include shame, embarrassment, gratitude and admiration.

Despite knowing that there's a strong correlation between physiological data with affect, authors identify a lack of studies directed towards social interaction, and the most studied non-verbal modalities are in fact facial expressions and tone for voice. This can be partially justified given that in conventional social situations, thees internal signals aren't so easily perceived. On this persuasion, two research directions are proposed, the first is to display social cues to foreign users through physiological data, The second involves analysing physiological activity among multiple users to facilitate collective interaction. The latter approach compliments \citeauthor{slovak_understanding_2012}'s proposal for \textit{composite signals} for HR as information (i.e., combining physiological signals from different people together) \cite{slovak_understanding_2012}.

% \textit{Connecting Brains and Bodies: Applying Physiological Computing to Support Social Interaction}
% Authors note a study by Levenson and Ruef which relates the physiological linkage between two participants with empathy .
% Physiological activity can be defined as a social cue, it can convey information about one's emotional state whilst it is not the primary function. For example, blushing as a result of high blood pressure.

% There are only a few systems that utilise physiological signals to preform social inferences, aside from those from Affective Computing research dedicated to emotional assessment.

% It's explained how activity from the central nervous system correlates to affective and cognitive states during social interaction, where social emotions include shame, embarrassment, gratitude and admiration and and example of a cognitive signal can be attention. These physiological changes from our autonomic nervous system can be considered highly reliable as the y are very difficult to control and mostly involuntary.

\section{Concluding Remarks}
\label{lit_review:conclusion}

In this work, we aim to develop and evaluate methods of sharing physiological activity this is abstracted from raw signals and without explicit associations to linguistic descriptors. Through visual, auditory and tactile feedback, we intend to generate new representations of the subject's inner state inferred from analysing physiological signals, which can then be transmitted to foreign users as a means of communication.

\begin{figure}[htbp]
	\centering
	\includegraphics[width=1.0\textwidth]{Chapters/Figures/Abstracted_Representations.png}
	\caption{Types of Outputs for Representing Physiological Activity}
	\label{fig:Abstracted_Representations}
\end{figure}

In figure \ref{fig:Abstracted_Representations}, we show a simplified plot of how physiological activity can be presented either before or after inference. The leftmost end of the spectrum leans towards the ideas discussed in section \ref{lit_review:biosignals_sharing}, where live biofeedback depicts the low-level features of the signal. The rightmost end of the spectrum describes what is achieved in cognitive-based emotion recognition systems, where the data is computationally analysed to produce emotional inferences for the user, manifested as a high-level descriptors (e.g fear, joy, surprise, etc...). We intend to demonstrate of middle-ground, for which it's not neccessary to determine a particular socio-affective inference, nor intended to display . To achieve this, we take inspiration from performative uses of embodied sensor data, introduced in section \ref{lit_review:biosignals_creativity}, enabling features of the signal to generate abstract representation, to be perceived by a foreign body.
% In this case, it's expected for the external user to make socio-affective inferences from the abstracted representation.

Deviating from the methods surveyed in this section, we would like to use this research opportunity to delve into the affordances of contemporary machine learning techniques as a means of generating new representations of biosensor data. We consider this initiative to be a novel contribution to the field given the impression that machine learning systems are commonly purposed to derive high-level emotional descriptions for emotion recognition tasks. To stay persistent with the thesis objectives, these generative representations should align with the interactional perspective in regards to the non-reductionist principles summarized in section x.x. The aim is not to detect a singular 'right' or 'true' emotion, but rather, to inspire expressive dialogue and emotional reflection \cite{hook_affective_2009}.
% At the point of writing, we consider this initiative to be a novel contribution to the field given the impression that machine learning systems are commonly designed for emotion recognition tasks, deriving high-level emotional descriptions from a stream of physiological sensor data.

Our incentive for adopting non-representational machine learning solutions in this context is to exhaust the output possibilities when mapping decoded sensor data of relatively low dimensionality, and embracing high-granularity results that can expose expressive nuances that may not be so salient in the raw data alone.

% In addition, we would like to study upon the idea that machine learning systems can improve the level of personalization in affective interactive systems compared to what is normally expected. To achieve this, we would like to develop upon the concepts proposed by Interactive Machine Learning research [http://research.gold.ac.uk/24757/].

In summary, we offer the following design principles that will be adopted into practical assessment: Communication mediums should not infer emotional states that hold a predetermined meaning; physiological sensor data elicits feedback, however the content of which should not directly mirror the raw biological functions that correspond; Users are granted the responsibility to explore the affordances of a system according their bodies and interpret the representations that are produced in that process; Sensor data alone is not a sufficient means of expression, rather considered a way of navigating the aesthetic functions informed by the user and their surroundings.
% \begin{itemize}
%     \item Communication mediums should not infer emotional states that hold a predetermined meaning.
%     \item Physiological sensor data initiates feedback, however the content of which should not directly mirror the raw biological functions that correspond.
%     \item Users are granted the responsibility to explore the affordances of a system according their bodies and interpret the representations that are produced in that process.
%     \item Sensor data alone is not a sufficient means of expression, rather considered a way of navigating between aesthetic functions informed by the user and their surroundings.
% \end{itemize}
