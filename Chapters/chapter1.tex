%!TEX root = ../template.tex
%%%%%%%%%%%%%%%%%%%%%%%%%%%%%%%%%%%%%%%%%%%%%%%%%%%%%%%%%%%%%%%%%%%
%% chapter1.tex
%% NOVA thesis document file
%%
%% Chapter with introduction
%%%%%%%%%%%%%%%%%%%%%%%%%%%%%%%%%%%%%%%%%%%%%%%%%%%%%%%%%%%%%%%%%%%

\typeout{NT FILE chapter1.tex}

\section{Background}
\label{sec:objectives}

\begin{itemize}
  \item Significance of non-verbal communication in social contexts
  \item Most common non-verbal cues are facial expression and tone of voice
  \item Affective Computing proposes the use of physiological activity to infer emotional states. However, these rely on linguistic descriptors
  \item There is an emerging area of work that considers the sharing of physiological activity, proving to be an effective method of enhancing connectedness. However, these studies tend to utilize raw data, disregarding  properties of the signal informative of the subject's socio-affective state.
  \item Aesthetics allow us to convey emotional meaning
\end{itemize}

\section{Research Questions}
\label{sec:research_questions}

\begin{itemize}
  \item How can we represent physiological activity in ways that are non-verbal?
  \item How can aesthetics be Incorporated into visuals, sound and actuation to articulate and express emotional content?
  \item What features can be extracted from each signal to gain information about the subject socio-affective state?
  \item What specific social and emotional cues can be conveyed in a social context?
\end{itemize}

\section{Thesis Structure} % (fold)
\label{sec:structure}

Chapter \ref{cha:technical_concepts} provides a background to the technical and theoretical concepts that form the foundation of thesis. This includes an overview of the different physiological signals commonly used in the field, along with fundamental concepts in Social Signal Processing and Somaestheitic design. 

In Chapter \ref{cha:lit_review}, we comment on some relevant literature covering the following topics, sharing biosignals, biosignals and creative practice, traditional affective computing approaches, and the intersection of social signal processing.

In Chapter \ref{cha:technical_contributions}, we will introduce and examine a set of hardware and software tools that were developed within the duration of the PhD and adopted for the purposes of augmenting nonverbal and collaborative interactions. This will range from specialized wearable devices for physiological data sensing to systems for processing and mapping this incoming data in meaningful ways. This will be followed by a series of applications that have been realised using these tools, described in Chapter \ref{cha:case_studies}. We will then evaluate our use-cases, taking data from user studies.  

% \printbibliography[heading=subbibliography, segment=\therefsegment, title={\bibname\ for chapter~\thechapter}]
