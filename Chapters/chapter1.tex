%!TEX root = ../template.tex
%%%%%%%%%%%%%%%%%%%%%%%%%%%%%%%%%%%%%%%%%%%%%%%%%%%%%%%%%%%%%%%%%%%
%% chapter1.tex
%% NOVA thesis document file
%%
%% Chapter with introduction
%%%%%%%%%%%%%%%%%%%%%%%%%%%%%%%%%%%%%%%%%%%%%%%%%%%%%%%%%%%%%%%%%%%

\typeout{NT FILE chapter1.tex}

\section{Introduction}
\label{sec:objectives}

\begin{itemize}
  \item Significance of non-verbal communication in social contexts
  \item Most common non-verbal cues are facial expression and tone of voice
  \item Affective Computing proposes the use of physiological activity to infer emotional states. However, these rely on linguistic descriptors
  \item There is an emerging area of work that considers the sharing of physiological activity, proving to be an effective method of enhancing connectedness. However, these studies tend to utilize raw data, disregarding  properties of the signal informative of the subject's socio-affective state.
  \item Aesthetics allow us to convey emotional meaning
\end{itemize}

\section{Research Questions}
\label{sec:research_questions}

\begin{itemize}
%   \item How can we represent physiological activity in ways that are non-verbal?
%   \item How can aesthetics be incorporated into visuals, sound and haptic mechanisms to articulate and express emotional content?
%   \item What features can be extracted from each signal to gain information about the subject socio-affective state?
%   \item What specific social and emotional cues can be conveyed in a social context?
    \item Why should embodied sensor technologies be used to mediate speechless dialogue?
    \item What mediums are capable of producing emotionally meaningful representations of biosignals, suitable for social intervention?
    \item How can aesthetics be incorporated into visuals, sound and haptic mechanisms to articulate and express emotional content?
    \item Adopting methods from modern Machine Learning and New Media practices, how do we encourage user empowerment and personalisation in Affective Systems?
\end{itemize}

Atypically from prevailing Affective Computing systems, the intention is not to infer or classify emotion, but rather create new opportunities for rich gestural exchange, unconfined to the verbal domain.

\section{Thesis Structure} % (fold)
\label{sec:structure}

\subsection{Introduction}
The current section establishes the motivation and major research questions for the thesis research.

\subsection{Chapter 1: Theoretical and Technical Concepts}

Chapter \ref{cha:technical_concepts} provides a background to the technical and theoretical concepts that form the foundation of thesis. To begin, we devide the proposed title into 4 constructs and establish the major themes of the research. This includes an overview of the different physiological signals commonly used in the field, along with fundamental concepts in Social Signal Processing and Somaestheitic design.

\subsection{Chapter 2: Literature Review}

In Chapter \ref{cha:lit_review}, we comment on some relevant literature covering the following key topics of interest: Affective Computing, Social Signal Processing, Machine Learning, Somaesthetic Design, biosignals in creative practice and sensor-based communication strategies.

\subsection{Chapter 3: Technical Contributions}

In Chapter \ref{cha:technical_contributions}, we will introduce and examine a set of hardware and software tools that were developed within the duration of the PhD and adopted for the purposes of augmenting nonverbal and collaborative interactions. This will range from specialized wearable devices for physiological data sensing to systems for processing and mapping this incoming data in meaningful ways. This will be followed by a series of applications that have been realised using these tools, described in Chapter \ref{cha:case_studies}. We will then evaluate our use-cases, taking data from user studies.  

\subsection{Chapter 4: Preliminary Actions}

Preliminary Actions

\subsection{Chapter 5: Major Case Studies}

We present a set of experimental systems developed during the PhD, each intended to establish a theoretical framework for designing systems for emotional exchange. Given this experimental methodology, a matured research space should provoke and sustain discussions around the socio-political implications of technological interventions and appropriation in preparation to be deployed in out-of-lab environments, eventually gearing towards pervasive engagement.

\subsection{Chapter 6: Results and Discussion}


% \printbibliography[heading=subbibliography, segment=\therefsegment, title={\bibname\ for chapter~\thechapter}]
