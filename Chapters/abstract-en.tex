%!TEX root = ../template.tex
%%%%%%%%%%%%%%%%%%%%%%%%%%%%%%%%%%%%%%%%%%%%%%%%%%%%%%%%%%%%%%%%%%%%
%% abstract-en.tex
%% NOVA thesis document file
%%
%% Abstract in English([^%]*)
%%%%%%%%%%%%%%%%%%%%%%%%%%%%%%%%%%%%%%%%%%%%%%%%%%%%%%%%%%%%%%%%%%%%

\typeout{NT FILE abstrac-en.tex}

To communicate historically implies the transfer of information between bodies, yet this phenomenon is constantly adapting to new technological and cultural standards. In a digital context, it’s commonplace to envision systems that revolve around verbal modalities. However, behavioural analysis grounded in psychology research calls attention to the emotional information disclosed by non-verbal social cues, in particular, actions that are involuntary. This notion has circulated heavily into various interdisciplinary computing research fields, from which multiple studies have arisen, correlating non-verbal activity to socio-affective inferences. These are often derived from some form of motion capture and other wearable sensors, measuring the ‘invisible’ bioelectrical changes that occur from inside the body. 

This thesis proposes a motivation and methodology for using physiological sensory data as an expressive resource for technology-mediated interactions. Initialised from a thorough discussion on state-of-the-art technologies and established design principles regarding this topic, then applied to a novel approach alongside a selection of practice works to compliment this. We advocate for aesthetic experience, experimenting with abstract representations. Atypically from prevailing Affective Computing systems, the intention is not to infer or classify emotion but rather to create new opportunities for rich gestural exchange, unconfined to the verbal domain. 

Given the preliminary proposition of non-representation, we justify a correspondence with modern Machine Learning and multimedia interaction strategies, applying an iterative, human-centred approach to improve personalisation without the compromising emotional potential of bodily gesture. Where related studies in the past have successfully provoked strong design concepts through innovative fabrications, these are typically limited to simple linear, one-to-one mappings and often neglect multi-user environments; we foresee a vast potential. In our use cases, we adopt neural network architectures to generate highly granular biofeedback from low-dimensional input data.

We present the following proof-of-concepts: Breathing Correspondence, a wearable biofeedback system inspired by Somaesthetic design principles; Latent Steps, a real-time auto-encoder to represent bodily experiences from sensor data, designed for dance performance; and Anti-Social Distancing Ensemble, an installation for public space interventions, analysing physical distance to generate a collective soundscape. Key findings are extracted from the individual reports to formulate an extensive technical and theoretical framework to expand on this topic. The projects first aim to embrace some alternative perspectives already established within Affective Computing research. From here, these concepts evolve deeper, bridging theories from contemporary creative and technical practices with the advancement of biomedical technologies.

% \begin{verbatim}
%     \abstractorder(<MAIN_LANG>):={<LANG_1>,...,<LANG_N>}
% \end{verbatim}
% \noindent to the customization area in the document preamble, e.g.,
% \begin{verbatim}
%     \abstractorder(de):={de,en,it}
% \end{verbatim}

% Palavras-chave do resumo em Inglês
\begin{keywords}
Embodied Interaction, Physiological Sensors, Affective Computing, Non-Verbal Communication, Human-Computer Interaction, Expressive Interaction, Machine Learning.
\end{keywords} 