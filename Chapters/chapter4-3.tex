\section[SenseFM]{An Alternative Protocol for Ubiquitous Sensing and Mobile Applications: \\ {\large \indent\hspace{0.09\textwidth} The Case for Sensory Audio and Mobile Web Frameworks}}

\subsection{Overview}

The past decade has in monuments advancements in smartphone technology development, offering continual benefits towards daily communication needs. However, the demands for mass production come at the costs of accumulating excessive e-waste for products that cannot keep up with a culture of impulsive consumption, despite being tremendously capable devices with a nearly ideal setup for portable sensing applications. The seminar outlines an alternative framework for acquiring sensor data by taking advantage of native networking and audio processing tools embedded into such consumer devices. This will provide a short background of existing practices in the domain, followed by a proposed alternative that in essence, bypasses specific dependencies on development libraries or additional applications. 

% We test this approach as part of an interactive installation, transmitting pervasive sensors that are situated in the streets of Mouraria, Lisbon. Finally, we provide insights parallel to state-of-the-art research in biomedical materials, opening up a path for the use of analog signals in modern mHealth applications.