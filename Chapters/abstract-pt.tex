%!TEX root = ../template.tex
%%%%%%%%%%%%%%%%%%%%%%%%%%%%%%%%%%%%%%%%%%%%%%%%%%%%%%%%%%%%%%%%%%%%
%% abstrac-pt.tex
%% NOVA thesis document file
%%
%% Abstract in Portuguese
%%%%%%%%%%%%%%%%%%%%%%%%%%%%%%%%%%%%%%%%%%%%%%%%%%%%%%%%%%%%%%%%%%%%

\typeout{NT FILE abstrac-pt.tex}

% \textcolor{red}{needs checking}
Historicamente, os processos de comunicação implicam a transferência de informação entre organismos, mas este fenómeno está constantemente a adaptar-se a novos padrões tecnológicos e culturais. Num contexto digital, é comum encontrar sistemas que giram em torno de modalidades verbais. Contudo, a análise comportamental fundamentada na investigação psicológica chama a atenção para a informação emocional revelada por sinais sociais não verbais, em particular, acções que são involuntárias. Esta noção circulou fortemente em vários campos interdisciplinares de investigação na área das ciências da computação, dos quais surgiram múltiplos estudos, correlacionando a actividade não-verbal com inferências sócio-afectivas. Estes são frequentemente derivados de alguma forma de captura de movimento e sensores ``wearable'', medindo as alterações bioeléctricas ``invisíveis'' que ocorrem no interior do corpo.

Nesta tese, propomos uma motivação e metodologia para a utilização de dados sensoriais fisiológicos como um recurso expressivo para interacções mediadas pela tecnologia. Iniciada a partir de uma discussão aprofundada sobre tecnologias de ponta e princípios de concepção estabelecidos relativamente a este tópico, depois aplicada a uma nova abordagem, juntamente com uma selecção de trabalhos práticos, para complementar esta. Defendemos a experiência estética, experimentando com representações abstractas. Contrariamente aos sistemas de Computação Afectiva predominantes, a intenção não é inferir ou classificar a emoção, mas sim criar novas oportunidades para uma rica troca gestual, não confinada ao domínio verbal.

Dada a proposta preliminar de não representação, justificamos uma correspondência com estratégias modernas de Machine Learning e interacção multimédia, aplicando uma abordagem iterativa e centrada no ser humano para melhorar a personalização sem o potencial emocional comprometedor do gesto corporal. Nos casos em que estudos anteriores demonstraram com sucesso conceitos de design fortes através de fabricações inovadoras, estes limitam-se tipicamente a simples mapeamentos lineares, um-para-um, e muitas vezes negligenciam ambientes multi-utilizadores; com este trabalho, prevemos um potencial alargado. Nos nossos casos de utilização, adoptamos arquitecturas de redes neurais para gerar biofeedback altamente granular a partir de dados de entrada de baixa dimensão.

Apresentamos as seguintes provas de conceitos: Breathing Correspondence, um sistema de biofeedback wearable inspirado nos princípios de design somaestético; Latent Steps, um modelo autoencoder em tempo real para representar experiências corporais a partir de dados de sensores, concebido para desempenho de dança; e Anti-Social Distancing Ensemble, uma instalação para intervenções no espaço público, analisando a distância física para gerar uma paisagem sonora colectiva. Os principais resultados são extraídos dos relatórios individuais, para formular um quadro técnico e teórico alargado para expandir sobre este tópico. Os projectos têm como primeiro objectivo abraçar algumas perspectivas alternativas às que já estão estabelecidas no âmbito da investigação da Computação Afectiva. A partir daqui, estes conceitos evoluem mais profundamente, fazendo a ponte entre as teorias das práticas criativas e técnicas contemporâneas com o avanço das tecnologias biomédicas.

% \begin{verbatim}
%     \abstractorder(<MAIN_LANG>):={<LANG_1>,...,<LANG_N>}
% \end{verbatim}
% \noindent à zona de customização no preâmbulo do documento, e.g.,
% \begin{verbatim}
%     \abstractorder(de):={de,en,it}
% \end{verbatim}

% Palavras-chave do resumo em Português
\begin{keywords}
Palavras-chave (em Português) \ldots
\end{keywords}
% to add an extra black line
