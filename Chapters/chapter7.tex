%!TEX root = ../template.tex
%%%%%%%%%%%%%%%%%%%%%%%%%%%%%%%%%%%%%%%%%%%%%%%%%%%%%%%%%%%%%%%%%%%%
%% chapter7.tex
%% NOVA thesis document file
%%
%% Chapter with lots of dummy text
%%%%%%%%%%%%%%%%%%%%%%%%%%%%%%%%%%%%%%%%%%%%%%%%%%%%%%%%%%%%%%%%%%%%

\typeout{NT FILE chapter7.tex}

% \chapter{Supplementary Essays: Speculative Developments, Theories and Criticisms of an Emerging Domain}

% One may wish to interpret this as a brief deviation from the scientific domain to something that is closely related to the humanities. With that said, this should not encourage a dualistic mono-disciplinary perspective between the two, but rather give an adequate margin of expression upon topics that are undeniably crucial yet conventionally undermined in shorter form publications, such as a conference paper or journal article.

\chapter{Closing Remarks}
\label{cha:conclusion}

In this final chapter, we summarise the outcomes in response to the major research goals described in Chapter \ref{cha:objectives} in addition to their corresponding publications serving as documentation these works, reviewed externally. We put fourth reflective insights gathered from three major case studies, framed into a cohesive narrative to be extended into future research efforts. Finally, we discuss some of the social implications to be accounted for when applying these theoretical frameworks to a new generation of communication technologies.

\section{Non-verbal Communication with Physiological Sensors}

In response to our major research questions that follow our initial motivation statement in Section \ref{sec:research_questions}, we declare the following outcomes of our research as a whole. Expanding upon previous works in this domain, we focus on the prospect of retrieving data of the subject's personal experience as part of the interaction itself that would in theory, allow interactive systems to be adaptable to its individuals and specific context of use, and to do so continuously to remain emotionally relevant.

\paragraph{Why should embodied sensor technologies be used to mediate speechless dialogue?}

Starting from the theoretical and technical concepts introduced in Section \ref{cha:technical_concepts}, we explain the importance of non-verbal behaviours during an interpersonal exchange, even when being acted and perceived unconsciously, these reveal various emotional cues that are not evident from speech alone. From here, we are interested in how our physiological activity, the internal changes that occur within the body could be appropriated as a tool for expressive mediation. Acknowledging the possibility to capture such information using embodied sensors, we look towards the interactive potentials of abstracting and transforming this data upon its raw numerical form. Finally, we make a resemblance with Somaesthetic theory to depict felt physical sensations according to one's personal experiences, further supported by a comprehensive literature review in Section \ref{cha:lit_review}, taking inspiration from previous studies from which we justify aesthetic function as a capable means for physiological representation.

\paragraph{What mediums are capable of producing emotionally meaningful representations of physiological signals, suitable for social intervention?}

A survey of relevant sensing and actuation technologies are presented as part of the preliminary research actions in Section \ref{cha:Preliminary_Actions_sens_act}, followed by three use cases in Chapters \ref{case_studies:soma_chi}, \ref{case_studies:latent_steps}, and \ref{case_studies:adse_ess}, each showcasing novel orchestrations used to assess visual, sound and haptic feedback mechanisms for real-time sensorial engagement.

We first uncover how shape-changing materials can be physically provocative when imposing unconventional movements onto the subject's limbs and organs, specifically when targeting the extremities engaged in the interaction at regulated intervals, forming a correspondence with the somatic nervous system. However, these systems revealed major drawbacks regarding portability between different spaces and users. Screen-based mediation opened up the possibilities of producing highly granular unique representations, leveraging proprioceptive sensibility and spatial aesthetics qualities described in Section \ref{cha:aestheitic_parameters}. Finally, working sound interaction arises a genuine viability to share feedback amongst multiple participants, capable of sustaining interpersonal dialogue and collective gesture, without disrupting one's other somatic resources, such as sight or touch. Additionally, mobile actuators along with rhythmic distribution strategies can maintain cultivate awareness to one's environment and surrounding behaviours.

In terms of sensing, we have documented our experiences with the following modalities: inertial motion (acceleration and orientation), proximity, respiration, and electromyography. Additionally, optical motion capture was incorporated in our second case study (Section \ref{case_studies:modi_dis}), accepting however, that this does not fit into the criteria of a wearable sensor per se. Other physiological signals of interest such as electrocardiography and electrodermal activity have merely been discussed in relation to previous works, for which we propose our findings can be transferable to using such inputs for an interactive system.

\paragraph{How can aesthetics be incorporated into visuals, sound and haptic mechanisms to articulate and express emotional content? And how do we encourage user empowerment and penalisation for effective intervention?}

The process of representation in non-verbal interaction has been routinely evaluated throughout the thesis, proposed as a way of conveying emotionally meaningful content by presenting identifiable traits from one's bodily activity to another person. We first describe representation in a theoretical manner, by which we perceiver and interpret new information based off of past experiences, then applying this to the design of novel mapping strategies. We offer three strategies for applying user-specified information from the user to context-aware emotional~modelling.

The first case study points focuses on the physical sensations aroused from haptic actuation materials according to user-controlled parameter adjustments, continuously responding to the conscious experiences of the individual at that given time. Following this, Latent Steps considers novel representations of physiological sensor sensor data, navigating non-linear patterns within several introspective interpretations as illustrated by the user during a self-reflective session for data collection; this was made possible with neural network modelling and conditioning of the latent space. Our final case study, focused on interpersonal communication, considers the social aspects of an encounter that are influenced by the way subjects coordinate themselves within a given space according to a distributed interaction architecture, and how one may feel about approaching such interactive environments, even without explicit invitation to do so.

Regarding empowerment, we recognise that flexible mapping strategies grants the user the final responsibility to determine what the system is inferring about their emotional experiences, as it has been expressed in the writings of \citeauthor{stahl_evocative_2014} \cite{stahl_evocative_2014}, and related works. Advancing from current human-centred modelling practices praising a heightened degree of user empowerment as a result of personalised mappings (e.g. \cite{carney_teachable_2020}), we embrace contributions from the user in order to curate unique training data resources. In addition, we propose that open inclusion and voluntary participation are crucial components when curating a valid sense of authorisation.

\section{Contributions}

\subsection*{List of publications}

\subsubsection*{Journal Papers}

\begin{itemize}
    \item[] \textit{Sound Feedback for Social Distance: The Case for Public Interventions during a Pandemic}
    \textbf{W. Primett}, H. Plácido da Silva, and H. Gamboa
    en. In: Electronics 11.14 July 2022, p. 2151. doi:
    \url{10.3390/electronics11142151}

    \item[] \textit{Biosensing and Actuation—Platforms Coupling Body Input-Output Modalities for Affective Technologies}
    M. Alfaras, \textbf{W. Primett}, M. Umair, C. Windlin, P. Karpashevich, N. Chalabianloo, D. Bowie, C. Sas, P. Sanches, K. Höök, C. Ersoy, and H. Gamboa
    en. In: \textit{Sensors} 20.21 (Jan. 2020), p. 5968. doi: \url{https://doi.org/10.3390/s20215968}
\end{itemize}

\subsubsection*{Conference Proceedings}
\begin{itemize}
    \item[] \textit{Designing Interactive Visuals for Dance from Body Maps: Machine Learning and
    Composite Animation Approaches}
    N. N. Correia, R. Masu, \textbf{W. Primet}t, S. Jürgens, J. Feitsch, and H. Plácido da Silva.
    In: \textit{Designing Interactive Systems Conference.}
    DIS ’22. New York, NY, USA, June 2022, pp. 204–216. doi: \url{10.1145/3532106.3533467}

    \item[] \textit{Exploring Awareness of Breathing through Deep Touch Pressure}
    A. Jung, M. Alfaras, P. Karpashevich, \textbf{W. Primett}, and K. Höök.
    en. In: \textit{Proceedings of the 2021 CHI Conference on Human Factors in Computing Systems.} Yokohama Japan, May 2021,
    pp. 1–15. doi: \url{10.1145/3411764.3445533}

    \item[] \textit{How do Dancers Want to Use Interactive Technology?: Appropriation and Layers of Meaning Beyond Traditional Movement Mapping}
    R. Masu, N. N. Correia, S. Jurgens, I. Druzetic, and \textbf{W. Primett}
    en. In: \textit{Proceedings of the 9th International Conference on Digital and Interactive Arts}. Braga Portugal, October 2019, pp. 1–9. doi: \url{10.1145/3359852.3359869}
\end{itemize}

\subsubsection*{Secondments}
  \begin{itemize}
    \item[] KTH (Stockholm, Sweden), Supervised by Professor Kristina Höök for project \textit{SomaSensing: First-person approach to design toolkits that raise emotional self-awareness}, January 4 to Febuary 4, 2019. Followed by workshop: \url{https://www.affectech.org/2019/02/6th-affectech-training-in-milan-italy-emotion-regulation-and-virtual-reality-based-biofeedback/}

    \item[] M-ITI (Madeira, Portugal), Supervised by Professor Nuno Correia for project \textit{Moving Digits: Augmented Dance for Engaged Audience}, May 2 to June 18, 2019. Followed by workshop: \url{https://movingdigits.eu/2019/05/moving-digits-tech-week-report/}
  \end{itemize}

\subsubsection*{Dissemination Activities}
    \begin{itemize}

    \item[] \textit{An Alternative Protocol for Ubiquitous Sensing and mHealth Applications, The Case for Sensory Audio and Mobile Web Frameworks}. Seminar Doctoral Program, FCT-UNL, Caparica, Portugal, 2019

    \item[] \textit{Latent Steps: Generative Bodily Animations from a Collection of Human Drawings and Physiological Data Interaction} Poster and Demo Session
    \textbf{W. Primett}
    International Conference on Dance Data, Cognition and Multimodal Communication. DDCMC’19, September 2019.
    \url{http://ddcmc19.blackbox.fcsh.unl.pt/programme/}

    \item[] \textit{``A Beautiful Glitch''}
        S. Rijmer, H. `, L. Vares, S. Jürgens, R. Masu and J. Feitsch, \textbf{W. Primett}
    Moving Digits Artistic Residency, Sõltumatu Tantsu Lava, Tallinn, August 2019. Performance, Talk and Video Presentation
    \url{https://movingdigits.eu/artistic-residency/}
    \end{itemize}

\section{Reflections}
\label{sec:reflections}

To complete our research outputs, we present a number of perspectives given the speculation and assumption that this is already a matured research space. Some of which are already put forward in the previous sections, but described more in-depth and unconfined from the current research outcomes of the thesis, which for now can be considered elementary.

\subsection*{The Case for Integrating Technology with Established Movement Practices}

Contemporary Interaction Design (IxD) research groups have incorporated Contact Improvisation routines into their work, enlightening new perceptions of mobility through a collaborative practice that involves the transfer of body weight between partners \cite{bomba_somacoustics_2019, barrero_gonzalez_dance_2019}. Where the principal activity presumes that intimate space is to be co-occupied and that touch is freely permitted, in some cases using intense pressure, such practices emphasise the importance of consent and appropriating of physical contact, though several practitioners have come forward to challenge this idealism \cite{tai_exploring_2017,beaulieux_how_2019}.

\subsection*{Managing Experienced Practice with Pervasiveness}

In the our initial pursuit towards integrating established somatic practices to the domain of sensory technologies, a two month research residency commenced with KTH, Royal Institute of Technology. The proposed outcome for this collaboration was ultimately to develop an Interactive Machine learning framework that models specific physiological characteristics informed by Contact Improvisation (CI) and Feldenkrais. To initiate our design process, a small series of workshop sessions, guided by an expert practitioner, serving as a somatic connoisseur. Between each activity, participants would highlight some of the aesthetic sensations that occurred, and try to develop sensor-actuation couplings that would mirror the inter-user movement qualities.

On a personal account, the intermediate CI workshops did not enlighten the prolific research opportunities that were naively anticipated up until the research residency. First off, we were not so clear as to the major role of technological intervention. For example, to guide movement patterns in substitute of a facilitator, or exaggerate tactile sensation that occur. In terms of inclusion, the experiential gap between the facilitator and the researchers felt disruptive to personal exploration. Even given a successful digitisation with embodied sensors, how would this experience be distributed outside of the lab, or even a studio? Moreover, prospects for guiding non-verbal communication only become more distant. Without dwelling much further, this segment concludes with the a call to new research directions that realised throughout our research outputs. Ultimately, we are interested in systems capable to instigate interaction without the necessity of explicit instructions. To essentially allow participant to experience the interactive artefact through experimentation, preferably shared with others.

\subsection*{Tensions Between Research and Artistic Creation}

One of the problems identified in the second case study (Sections\ref{case_studies:modi_dis}) was the lack of time to adequately develop artistic ideas with the technology.%, namely by C2 in Stage 4.
This reflects a tension between the time and budget available for academic research (in this case, involving professional artists, recruited through an open call, being rewarded for their participation in the project), and the time and budget needed to develop a performance in the intersection of contemporary dance and interactive technology. Although some positive aspects came out of this tension, such as finding functionalities in our systems that would speed up the workflow (Section \ref{subsec:modi_worksflow_c2}), or strategies to counterbalance the slow responsiveness of a prototype (Section \ref{subsec:modi_responsiveness_c1}), efforts should be made to attenuate time tensions when involving artists. A possible solution could be to consult with dance artists already when preparing the research plan and budget, to obtain advice regarding the adequacy of the time planned for artistic development, and what trade-offs to apply if needed.

In this particular study, there is also a risk of bias due to the fact that the study participants were professional dancers, and rewarded as such by our research project in terms of an artist fee. The fact that our research team worked closely with the group of participants for a long time, leading to familiarity and sympathy, may also have induced bias. As researchers, we were leading the organisation of the project, its activities and aims. Therefore, we might have introduced additional bias, as there might have been a perception of a power shift toward our research team. These elements could potentially inhibit some harsher criticism. However, we believe we mitigated that risk by stating repeatedly at each stage that all feedback, positive or negative, was important to improve the research being conducted.

\subsection*{Validity in Self Studies and Non-lab Environments}

The deliverables of the thesis carry upon theoretical underpinnings grounded in complex movement and bodily practices, namely Yoga and contemporary dance performance. It should be noted that the self-studies carried out during the prototyping in the evaluation stages lack the expert opinions of movement specialists, which would require inclusion professional practitioners. While study subjects were able to admit to holding some valuable experience, at least in complimentary practices, we openly the acknowledge having a limited understanding of the rich intricacies that lie in such specialist areas.

In the case of the research actions that called upon specialist users (e.g Preliminary Actions II, Chapter \ref{preliminary_actions:modi_ws1}), our results can be praised for interdisciplinary inclusion, for which we benefited from gaining alternative perspectives of the given technologies. That said, this approach without a doubt compromises on longitudinal prospects, imposed by time and budget contingencies. In our case, we found that it was more difficult to obtain highly structured data from these user studies, as we embraced more an exploration process of the experiential affordances, proceeding to outcomes in the form of interviews and focus groups. We found many difficulties to come against the scientific rigour seen in clinical trials, and the possibility to validate numerical findings. In reflection of the thesis outcomes, we construct a multi-stage research methodology that first embraces the incorporation of specialist users during an intensive preparation period, supporting longer-term studies that can be safely situated “in-the-wild” when ready. In public space, the participants are not burdened by being critically observed, their behaviour, even if not aligned with the study protocol, is authentic. They are responsible for using a system according to their personal intuition, not necessarily what the designer intended. Unlike lab trials, public space experimentation lends itself to unforeseeable events. Every encounter is unique in a way that cannot be perfectly repeated.

\section{Future Work}

\subsection*{Extending Individual Works}

% \subsection{Communication through Shape Changing}
Our Case Study I, Chapter \ref{case_studies:soma_chi} takes a look at shape changing materials for respiratory feedback, taking an introspective account of aesthetic sensation. While the results are not directed toward social communication, we know that learning to control our own breathing apparatus may also lead to better control over what we communicate to others. For example, consciously breathing in rhythm with someone else can lead to better communication and empathy \cite{keller_rhythm_2014}. Given that this investigation was incentivised by the experiments described in Section \ref{sec:breathing_synchrony}, uncovering motivations for sound-based actuation and physiological synchrony, we foresee potentials for shared tactile engagment also.
 % as part of social interaction environments.

% \subsection{Humanizing Procedural Represenations}
Our Case Study II, Chapter \ref{case_studies:modi_dis} compares two strategies for generating somatic representations, one procedural (MLIV) and the other produced manually (CAIV). We hypothesise that both approaches can be complementary, and combining both could lead to a ‘third way’. This ‘third way’ could also be adequate for free-form visualisation: where the human interpretation simplifies and harmonises the raw visual data, resulting in animation frames that are then used to train a generative machine learning model. This would combine the humanised and procedural qualities identified in these two adopted approaches.
On the technical side, our Section \ref{subsec:alternative_ml} proposes benefits for a further investigation into alternative machine learning architectures and data collection strategies. Though acknowledging that such technical efforts should be adequately balanced with steps for improving the user's comprehension of the process.
% Additional work in this area may also involve evaluating further these systems ‘in the wild’, in public performances, to assess if audiences value these visualisations of the dancers.

% \subsection{Hybrid Sensing Networks for Public Participation}
In our Case Study III, Chapter \ref{case_studies:adse_ess}, we outline the standout progressions between the two sensing mechanisms, one situated directly onto the body, the other installed into the surroundings, proving transferable qualities amidst persisting limitations when subjected to the general public. For future work, we foresee the benefit of incorporating a hybrid system comprised of wearable and environmental sensors, suitable for large open spaces in the confidence of robust operation. From here, we also look towards long-term studies with diverse user groups, crucial in forming generalisable conclusions of social behaviour with interpersonal feedback strategies.

\subsection*{Designing for Emotional Availability}

% The research outcomes of the thesis contained many learning points that were neglected in the scope of the initial proposal stage, formed over 4 years since its completion.
To construct speculations for future works, we would like to reflect on the importance of spontaneity, comfort and familiarity as part of the (non-verbal) communication process. Familiarity has been long-established to play a pinnacle role in general applications of interaction design \cite{dix_starting_1998}, accepted as a method of empowering users to routinely execute functions without deliberate thought processing. In everyday life, the non-sensory clothing and accessories that one wears cultivates an aesthetic engagement with the surroundings, not necessarily in a solitary state, but while being perceived as part of someone's overall appearance, and contextualised within a social scenario. In this manner, perhaps it's desirable to shift our perspective away from wearable technology to the normalisation of expressive materials that also happen to be embedded with sensors and actuators, as it is becoming the case with other constituents of modern urban environments, encompassing new aesthetic characteristics in our everyday life. For example, in architecture \cite{alvarez_re-imagining_2017}, mobility \cite{nesmachnow_bus_2020}, and fashion \cite{buruk_snowflakes_2021,bang__olufsen_press_collaboration_2022}. Where there exists a mutual entanglement bet wen technology and physical space, \citeauthor{kitchin_codespace_2011} explain how it's possible to reduce the presence of computation to the extent of conceiving augmented environments, where by our experience of the world is conditioned by digital articulations \cite{kitchin_codespace_2011}. 

The emergence of 'invisible' technologies showcase a viable alternative for traditional wearable sensors, enabling users to bypass the voluntary procedures of contacting electrodes with their skin in a specific manner \cite{dos_santos_silva_design_2021}. Another highly promising research path for ubiquitous sensing technologies can be seen in smart garments and e-textiles \cite{jarusriboonchai_customisable_2019}. One example in particular, recently exposed to academic literature is the development of transducer-embedded fabrics, achieving pervasive physiological monitoring when fitted into sensory garment \cite{yan_single_2022}. Given that such innovations are able to be \textit{``woven into everyday life''} \cite{song_smart_2022}, we insist that future works should prioritise the use of non-invasive sensing technologies to facilitate natural interaction in new environments, to preserve the ubiquitous nature of street-embedded technology without compromising the physiological fidelity that comes personal embodied technologies, and to coexists with familiarised spaces and practices. In this circumstance, the garment that one wears unconditionally truly serves as a social mediator, without being expected to instigate or characterise the interaction from the beginning.

\section{Designing Technologies for Expressive, Speechless Dialogue}
% \textcolor{red}{I don't think this conclusion says anything new, but maybe that's the point}

Over 7 chapters, we document a thorough enquiry into potential communication strategies with wearable sensors. The process of which took place from July 2018 to August 2022
, thus compiling just over 4 years of investigation. This emerges from a relevant understanding of aesthetic experience as emotional engagement, impartial to linguistic descriptors in a way that can be interpreted from third-persons. This develops into a generalisable criteria that appeals to two fundamental parameters, rhythm and space, applied to the practice representing the physical sensations derived from embodied sensor data, orchestrating various digital mediums and modalities. These concepts are realised into a series of orchestrations, taking upon the following scenarios, starting from self-experimentation, then into live performance, involving specialist users and audience perspectives, and eventually public spaces, commentating on the role of ubiquitous sensing technologies to facilitate spontaneous communication between participants non-acquainted with one another. This offers an alternative perspective on affective technologies, with the interest of developing one's sense of belongingness alongside their community, and sustaining long-term wellbeing. To conclude, we defend our initial research proposal, for which non-verbal communication can be considered an expressive resource that inspires novel social situations to arise as part of one's daily routine, and therefore asserting aesthetic consideration at the forefront of technologies for expressive, speechless dialogue.

\begin{quote}
\textit{``More than mathematical mathematical machines, computers are linguistic machines~that happen to be particularly effective at manipulating numbers. At the same time ,they treat language, as they treat numbers, login, and pretty much everything else, 'with an odd combination of practicality and philosophical abstraction' that inevitably shapes our own human language. For humans and computers, language operates in the liminal space between reality, its description, and its construction. human consciousness runs on information and develops a simulacrum, a virtual reality that allows us to negotiate with the external reality, to monitor our bodies, predict our behaviours and those of others. For this we use representational systems and signifies as standings for other things, working in their absence, allowing memory and imagination to call to mind things that are either not present or do not exist at all.''} \citeauthor{carvalhais_art_2022} \cite{carvalhais_art_2022}
\end{quote}
% TODO add internal references

% \subsection{Interactional Proxemics}
% As is the case with all kinds of non-verbal behaviours, multi-user proxemic interaction is a complex and an inherently ambiguous process. Without constraining the expressive freedoms that would be granted I everyday settings, a real-time biofeedback system should reflect on this. Though, how do we produce meaningful representations when the raw sensor data is desperately limited in dimensionality and temporal scale? Even more so than data from inertial motion sensors  or indeed motion capture.

% We already point out the research scarcity for interactional proxemics, continuously outweighed by studies that favour asynchronous data analysis, not interfering with the interaction space itself.  This is pretty much customary throughout Behavioural Computing research, though we may raise the question of why the divide is so intensely divided in this case. Does this simply   denote a novel design space that is lacking structural support from academia, or could the experimental reluctance be a consequence of technical challenges specific to the sensing technology?

% \subsection{Combining Sensor Data from Multiple Users}

% The essence of proxemics implies the use of multimodal information. Our system does not take an explicit measure of orientation, however, we can combine the relative proxemic data from all the sensors in order to distinguish when two or more users are facing towards one another, and register this as periods of mutual gaze. This pluralist gesture reveals a lot about the situation. Generally, we may discern moments of intentional exchange and affirmation, whether that in a comforting or confrontational manner. Of course, this data does not reveal the entire social context by any means, but in contrast to one user staring at the back of the back of another, given the same radial distance, implies a totally different dynamic. In our system, the sonic representations are designed to emphasise periods of interpersonal gaze. This is detected when the pulse signal from two or more sensors start interfering with one another.
