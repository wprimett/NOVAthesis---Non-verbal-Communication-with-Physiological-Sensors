%!TEX root = ../template.tex
%%%%%%%%%%%%%%%%%%%%%%%%%%%%%%%%%%%%%%%%%%%%%%%%%%%%%%%%%%%%%%%%%%%%
%% chapter7.tex
%% NOVA thesis document file
%%
%% Chapter with lots of dummy text
%%%%%%%%%%%%%%%%%%%%%%%%%%%%%%%%%%%%%%%%%%%%%%%%%%%%%%%%%%%%%%%%%%%%

\typeout{NT FILE chapter7.tex}

% \chapter{Supplementary Essays: Speculative Developments, Theories and Criticisms of an Emerging Domain}

% \label{cha:supplementary_essays}

% To complete our research outputs, we present a number of perspectives given the speculation and assumption that this is already a matured research space. Some of which are already put forward in the previous sections (\ref{cha:technical_contributions} and \ref{cha:lit_review} in particular), but described more in-depth and unconfined from the current research outcomes of the thesis, which for now can be considered elementary. These are formatted as a series of short essays purposed to shine light upon the socio-political implications of systems that take on our design methodology.

% One may wish to interpret this as a brief deviation from the scientific domain to something that is closely related to the humanities. With that said, this should not encourage a dualistic mono-disciplinary perspective between the two, but rather give an adequate margin of expression upon topics that are undeniably crucial yet conventionally undermined in shorter form publications, such as a conference paper or journal article.

% As the last chapter of this thesis, we present a summary of the work developed in line with the main objectives described in Chapter 1. The outcomes are further detailed, as well as the publications released. Then, generally looking to our results, we describe possible applications in which the outcomes of this thesis may be used. Some suggestions to extend the results of this work are presented in future work. To conclude, the implications of the attained work, in terms of the acquisition and analysis of human behaviour, are discussed. [COPIED]

\chapter{Reflections, Discussion and Conclusion}
\label{cha:conclusion}

In this final chapter, we summarise the outcomes in response to the major research goals described in Chapter \ref{cha} in addition to their corresponding publications serving as documentation these works, reviewed externally. We put fourth reflective insights gathered from three major case studies, framed into a cohesive narrative to be extended into future research efforts. Finally, we discuss some of the social implications to be accounted for when applying these theoretical frameworks to a new generation of communication technologies.

\section{Contributions}

\subsection{List of publications}

\subsubsection{Journal Papers}

\begin{itemize}
    \item[] \textit{Sound Feedback for Social Distance: The Case for Public Interventions during a Pandemic}
    \textbf{W. Primett}, H. Plácido da Silva, and H. Gamboa
    en. In: Electronics 11.14 July 2022, p. 2151. doi:
    \url{10.3390/electronics11142151}

    \item[] \textit{Biosensing and Actuation—Platforms Coupling Body Input-Output Modalities for Affective Technologies}
    M. Alfaras, \textbf{W. Primett}, M. Umair, C. Windlin, P. Karpashevich, N. Chalabianloo, D. Bowie, C. Sas, P. Sanches, K. Höök, C. Ersoy, and H. Gamboa
    en. In: \textit{Sensors} 20.21 (Jan. 2020), p. 5968. doi: \url{https://doi.org/10.3390/s20215968}
\end{itemize}

\subsubsection{Conference Proceedings}
\begin{itemize}
    \item[] \textit{Designing Interactive Visuals for Dance from Body Maps: Machine Learning and
    Composite Animation Approaches}
    N. N. Correia, R. Masu, \textbf{W. Primet}t, S. Jürgens, J. Feitsch, and H. Plácido da Silva.
    In: \textit{Designing Interactive Systems Conference.}
    DIS ’22. New York, NY, USA, June 2022, pp. 204–216. doi: \url{10.1145/3532106.3533467}

    \item[] \textit{Exploring Awareness of Breathing through Deep Touch Pressure}
    A. Jung, M. Alfaras, P. Karpashevich, \textbf{W. Primett}, and K. Höök.
    en. In: \textit{Proceedings of the 2021 CHI Conference on Human Factors in Computing Systems.} Yokohama Japan, May 2021,
    pp. 1–15. doi: \url{10.1145/3411764.3445533}

    \item[] \textit{How do Dancers Want to Use Interactive Technology?: Appropriation and Layers of Meaning Beyond Traditional Movement Mapping}
    R. Masu, N. N. Correia, S. Jurgens, I. Druzetic, and \textbf{W. Primett}
    en. In: \textit{Proceedings of the 9th International Conference on Digital and Interactive Arts}. Braga Portugal, October 2019, pp. 1–9. doi: \url{10.1145/3359852.3359869}
\end{itemize}

\subsubsection{Dissemination Activities}
    \begin{itemize}

    \item[] \textit{An Alternative Protocol for Ubiquitous Sensing and mHealth Applications, The Case for Sensory Audio and Mobile Web Frameworks}. Seminar Doctoral Program, FCT-UNL, Caparica, Portugal, 2019

    \item[] \textit{Latent Steps: Generative Bodily Animations from a Collection of Human Drawings and Physiological Data Interaction} Poster and Demo Session
    \textbf{W. Primett}
    International Conference on Dance Data, Cognition and Multimodal Communication. DDCMC’19, September 2019.
    \url{http://ddcmc19.blackbox.fcsh.unl.pt/programme/}

    \item[] \textit{"A Beautiful Glitch"}
    S. Rijmer, H. Junti, L. Vares, S. Jürgens, R. Masu and J. Feitsch, \textbf{W. Primett}
    Moving Digits Artistic Residency, Sõltumatu Tantsu Lava, Tallinn, August 2019. Talk and Video Presentation
    \url{https://movingdigits.eu/artistic-residency/}
    \end{itemize}

\section{Reflections}

\subsection{Aesthetic Representation as Non-verbal Modality}

The process of representation in non-verbal interaction has been routinely evaluated throughout the thesis, proposed as a way of conveying emotionally meaningful content by presenting identifiable traits from one's bodily activity to another person. We first describe representation in a theoretical manner, by which we perceiver and interpret new information based off of past experiences, then applying this to the design of novel mapping strategies.

- Parameter Adjustment
- Neural Network Modelling
- Distributed Interaction Architectures

Context aware
Context comes as a crucial part of emotional modelling

% \subsection{Interactional Proxemics}
% As is the case with all kinds of non-verbal behaviours, multi-user proxemic interaction is a complex and an inherently ambiguous process. Without constraining the expressive freedoms that would be granted I everyday settings, a real-time biofeedback system should reflect on this. Though, how do we produce meaningful representations when the raw sensor data is desperately limited in dimensionality and temporal scale? Even more so than data from inertial motion sensors  or indeed motion capture.

% We already point out the research scarcity for interactional proxemics, continuously outweighed by studies that favour asynchronous data analysis, not interfering with the interaction space itself.  This is pretty much customary throughout Behavioural Computing research, though we may raise the question of why the divide is so intensely divided in this case. Does this simply   denote a novel design space that is lacking structural support from academia, or could the experimental reluctance be a consequence of technical challenges specific to the sensing technology?

% \subsection{Combining Sensor Data from Multiple Users}

% The essence of proxemics implies the use of multimodal information. Our system does not take an explicit measure of orientation, however, we can combine the relative proxemic data from all the sensors in order to distinguish when two or more users are facing towards one another, and register this as periods of mutual gaze. This pluralist gesture reveals a lot about the situation. Generally, we may discern moments of intentional exchange and affirmation, whether that in a comforting or confrontational manner. Of course, this data does not reveal the entire social context by any means, but in contrast to one user staring at the back of the back of another, given the same radial distance, implies a totally different dynamic. In our system, the sonic representations are designed to emphasise periods of interpersonal gaze. This is detected when the pulse signal from two or more sensors start interfering with one another.

\subsection{The Case for Integrating Technology with Established Movement Practices}

Contemporary Interaction Design (IxD) research groups have incorporated Contact Improvisation routines into their work, enlightening new perceptions of mobility through a collaborative practice that involves the transfer of body weight between partners \cite{bomba_somacoustics_2019, barrero_gonzalez_dance_2019}. Where the principal activity presumes that intimate space is to be co-occupied and that touch is freely permitted, in some cases using intense pressure, such practices emphasise the importance of consent and appropriating of physical contact, though several practitioners have come forward to challenge this idealism \cite{tai_exploring_2017,beaulieux_how_2019}.

In the our initial pursuit towards integrating established somatic practices to the domain of sensory technologies, a two month research residency commenced with KTH, Royal Institute of Technology. The proposed outcome for this collaboration was ultimately to develop an Interactive Machine learning framework that models specific physiological characteristics informed by Contact Improvisation (CI). This involved a small series of workshop sessions, guided by an expert practitioner, serving as a somatic connoisseur. Between each activity, participants would highlight some of the aesthetically informed sensations that occurred, and try to develop sensor-actuation couplings that would mirror the inter-user movement qualities.

\subsection{Managing Complexity with Pervasiveness}

On a personal account, the intermediate CI workshops did not enlighten the prolific research opportunities that were naively anticipated up until the research residency. First off, we were not so clear as to the major role of technological intervention. For example, to guide movement patterns in substitute of a facilitator, or exaggerate tactile sensation that occur. In terms of inclusion, the experiential gap between the facilitator and the researchers felt disruptive to personal exploration. Even given a successful digitisation with embodied sensors, how would this experience be distributed outside of the lab, or even a studio? Moreover, prospects for guiding non-verbal communication only become more distant. Without dwelling much further, this segment concludes with the a call to new research directions that realised throughout our research outputs. Ultimately, we are interested in systems capable to instigate interaction without the necessity of explicit instructions. To essentially allow participant to experience the interactive artefact through experimentation, preferably shared with others.

\subsection{Spontaneity, Comfort and Familiarity}

The research outcomes of the thesis contained many learning points that were neglected in the scope of the initial proposal stage, formed over 4 years since its completion. We would like to reflect on some additional matters, the importance of spontaneity, comfort and familiarity as part of the (non-verbal) communication process.

\section{Limitations}
\label{sec:limitations}

\subsection{Self Studies and Non-lab Experimentation Settings}

The deliverables of the thesis carry upon theoretical underpinnings grounded in complex movement and bodily practices, namely Yoga and contemporary dance performance. It should be noted that the self-studies carried out during the prototyping in the evaluation stages lack the expert opinions of movement specialists, which would require inclusion professional practitioners. While study subjects were able to admit to holding some valuable experience, at least in complimentary practices, we openly the acknowledge having a limited understanding of the rich intricacies that lie in such specialist areas.

In the case of the research actions that called upon specialist uses (e.g Workshops 1.x, CSL Artistic Residency), our results can be praised for interdisciplinary inclusion, for which we benefited from gaining alternative perspectives of the given technologies. That said, this approach without a doubt compromises on longitudinal prospects, imposed by time and budget contingencies. In our case, we found that it was more difficult to obtain highly structured data from these user studies, as we embraced more an exploration process of the experiential affordances, proceeding to outcomes in the form of interviews, focus groups, etc…

We found many difficulties to come against the scientific rigour seen in clinical trials, and the possibility to validate numerical findings. In reflection of the thesis outcomes, we construct a multi-stage research methodology that first embraces the incorporation of specialist users during an intensive preparation period, supporting longer-term studies that can be safely situated “in-the-wild” when ready. In public space, participants are not being observed, their behaviour, even if not aligned with the study protocol, is authentic. They are responsible for using a system according to their personal intuition, not necessarily what the designer intended. Unlike lab trials, public space experimentation lends itself to unforeseeable events. Every encounter is unique in a way that cannot be perfectly repeated.

\subsubsection{Tensions Between Research and Artistic Creation}

One of the problems identified in case study 2 (Sections\ref{case_studies:modi_dis}) was the lack of time to adequately develop artistic ideas with the technology, namely by C2 in Stage 4. This reflects a tension between the time and budget available for academic research (in this case, involving professional artists, recruited through an open call, being rewarded for their participation in the project), and the time and budget needed to develop a performance in the intersection of contemporary dance and interactive technology. Although some positive aspects came out of this tension, such as finding functionalities in our systems that would speed up the workflow (in the case of C2), or strategies to counterbalance the slow responsiveness of a prototype (in the case of C1), efforts should be made to attenuate time tensions when involving artists. A possible solution could be to consult with dance artists already when preparing the research plan and budget, to obtain advice regarding the adequacy of the time planned for artistic development, and what trade-offs to apply if needed.

In this particular study, there is also a risk of bias due to the fact that the participants were professional dancers, and rewarded as such by our research project in terms of an artist fee. The fact that our research team worked closely with the group of participants for a long time, leading to familiarity and sympathy, may also have induced bias. As researchers, we were leading the organization of the project, its activities and aims. Therefore, we might have introduced additional bias, as there might have been a perception of a power shift toward our research team. These elements could potentially inhibit some harsher criticism. However, we believe we mitigated that risk by stating repeatedly at each stage that all feedback, positive or negative, was important to improve the research being conducted.

% \subsection{Inclusiveness and Pluralism}

% Progressive HCI researchers continue to strive towards technological solutions deemed universally appropriate  for neglected user groups, and opening fruitful discussions in doing so. Though, there still exists a certain benchmark of academia, if we take into account the heavily eurpeoan-american centring persuasion of published material.

% For a stronger insight into such progressions, we would prefer to point readers to the resources arising from other research communities, ballroom dance (vogue), ..., for example.

\section{Future Works}

\subsection{Designing for Emotional Availability and Authentic Interaction}

The emergence of 'invisible' sensing technologies showcase a viable alternative for traditional wearable sensors, enabling users to bypass the voluntary procedures of contacting electrodes with their skin in a specific manner.

% A garment that one would wear anyway, as a social mediator, an artefact should not an instigator of the interaction
The (non-sensory) clothing and accessories that we wear everyday provide an aesthetic engagement with our surroundings, not necessarily in their solitary state, but while being perceived as part of someone's overall appearance, and contextualised within a social scenario. In this manner, perhaps it's desirable to shift our perspective away from wearable technology to the normalisation of materials that also happen to be embedded with sensory capabilities, as it is becoming the case with other constituents of modern urban living, in achicture [], navigation [], and high-fasion [], for example.

To preserve the ubiquitous nature of street-embedded technology without comprimising the physiological fidenlioty that comes personal embodied technologies, and to coexsist with familerized spaces and practices.

\section{conclusion: Designing Technologies for Expressive, Speechless Dialogue}

Over 7 chapters, we have documented a thorough enquiry into potential communication strategies with wearable sensors. The process of which took place from July 2018 to August 2022
, thus compiling just over 4 years of investigation. This emerges from a relevant understanding of aesthetic experience as emotional engagment, impartial to linguistic descptors. We
